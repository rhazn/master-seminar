\documentclass[12pt, notheorems]{beamer}
%\documentclass[12pt, notheorems, handout]{beamer}
\usepackage[utf8]{inputenc}
\usepackage{pgfpages}
\usepackage{amsmath}
\usepackage{amsthm}
\usepackage{amssymb}
\usepackage{dsfont}
\usepackage{tikz}
\usepackage{pgfplots}
\usepackage{float}
\usepackage{stmaryrd}

\usepackage{enumitem}

% TODO QUESTION: do we need to discuss this in ore detail? faithful assignment here:

% ANSWER: Das ist eine gute Frage. So wie ich das Paper von Booth und Meyer im
% Kopf habe werden dort die Ordnung mit epistemischen Zuständen
% gleichgesetzt (was Darwiche und Pearl ja nicht machen). Von daher halte
% ich das für ok es nicht im aller formalen Ausführlichkeit zu machen.
% Aber man könnte es im einen Satz erwähnen

% Representation theorem
% In mathematics, a representation theorem is a theorem that states that every abstract structure with certain properties is isomorphic to another (abstract or concrete) structure.
% In mathematics, an isomorphism is a structure-preserving mapping between two structures of the same type that can be reversed by an inverse mapping.

% Operationalization
% In research design, especially in psychology, social sciences, life sciences and physics, operationalization or operationalisation is a process of defining the measurement of a phenomenon that is not directly measurable, though its existence is inferred by other phenomena.
% In a broader sense, it defines the extension of a concept—describing what is and is not an instance of that concept.

\pgfplotsset{compat=1.17}

\setitemize{label=\usebeamerfont*{itemize item}
  \usebeamercolor[fg]{itemize item}
  \usebeamertemplate{itemize item}}

\newcommand{\modelsOf}[1]{\llbracket #1 \rrbracket}

\usefonttheme{professionalfonts}
\usetheme{Madrid}
\usecolortheme{orchid}

\setbeamertemplate{caption}[numbered]
\setbeamertemplate{theorems}[numbered]

% Unique numbering scheme for definitions and theorems with the "notheorems" options on loading beamer
% see https://tex.stackexchange.com/questions/82415/beamer-different-numbering-for-theorems-examples-definition-and-lemma
\makeatletter
    \ifbeamer@countsect
      \newtheorem{theorem}{\translate{Theorem}}[section]
    \else
      \newtheorem{theorem}{\translate{Theorem}}
    \fi
    \newtheorem{corollary}{\translate{Corollary}}
    \newtheorem{fact}{\translate{Fact}}
    \newtheorem{lemma}{\translate{Lemma}}
    \newtheorem{problem}{\translate{Problem}}
    \newtheorem{solution}{\translate{Solution}}

    \theoremstyle{definition}
    \newtheorem{definition}{\translate{Definition}}
    \newtheorem{definitions}{\translate{Definitions}}

    \theoremstyle{example}
    \newtheorem{example}{\translate{Example}}
    \newtheorem{examples}{\translate{Examples}}


    % Compatibility
    \newtheorem{Beispiel}{Beispiel}
    \newtheorem{Beispiele}{Beispiele}
    \theoremstyle{plain}
    \newtheorem{Loesung}{L\"osung}
    \newtheorem{Satz}{Satz}
    \newtheorem{Folgerung}{Folgerung}
    \newtheorem{Fakt}{Fakt}
    \newenvironment{Beweis}{\begin{proof}[Beweis.]}{\end{proof}}
    \newenvironment{Lemma}{\begin{lemma}}{\end{lemma}}
    \newenvironment{Proof}{\begin{proof}}{\end{proof}}
    \newenvironment{Theorem}{\begin{theorem}}{\end{theorem}}
    \newenvironment{Problem}{\begin{problem}}{\end{problem}}
    \newenvironment{Corollary}{\begin{corollary}}{\end{corollary}}
    \newenvironment{Example}{\begin{example}}{\end{example}}
    \newenvironment{Examples}{\begin{examples}}{\end{examples}}
    \newenvironment{Definition}{\begin{definition}}{\end{definition}}
\makeatother

% Notes
% Talk using pympress: brew install pympress
% Talk command: pympress -t 40 -n right ~/Documents/master/master-seminar/presentation/1901_3230880_philip_heltweg_presentation.pdf
\setbeameroption{show notes on second screen=right} % Both

% Coloring note-pages
\setbeamertemplate{note page}{\pagecolor{yellow!15}\insertnote\vfill\insertframenumber}
\setbeamertemplate{note page}{
	\pagecolor{yellow!15}
	\medskip
	SLIDE: \insertframenumber\\
	TITLE: \insertframetitle,

	\insertnote
}

\AtBeginSection[]
{
  \begin{frame}
    \frametitle{Table of Contents}
    \tableofcontents[currentsection]
  \end{frame}
}

\begin{document}
\title{Of judges, aliens and total preorders}
\author[Heltweg, Philip]{Heltweg, Philip\\pheltweg@gmail.com}
\institute{University of Hagen}
\date{March 13th, 2021}

\frame{\titlepage}
\note{
    \begin{itemize}
        \item Paper "How to Revise a Total Preorder" von booth and meyer
        \item Das papier bespricht vieles, wir konzentrieren uns auf: operatoren die die weltanschauung eines agenten unter neuen informationen anpassen
        \item Folien sind nummeriert unten rechts, bei Fragen am besten Nummer aufschreiben
    \end{itemize}
}

\begin{frame}{Table of Contents}
    \tableofcontents

    \note{
        \begin{itemize}
            \item Einführung mit Motivation, Forschungskontext und Beispiel
            \item Formaler Hintergrund: Festlegen Notation, was sind TPOs überhaupt
            \item Dann: Wie bauen Booth und Meyer auf der vorherigen Arbeit auf um tpos zu ändern
            \item BM operatoren: Wie benutzen BM die vorher besprochenen Metadaten, Definition von Eigenschaften ihrer Operatoren
            \item Iteration: Ausblick auf mehr als einen Änderungsschritt + konkreter Operator
        \end{itemize}
    }
\end{frame}

\section{Introduction}
\note{
    \begin{itemize}
        \item Jetzt: Verstehen der Forschungsfrage, Einordnung des Papiers in einen Forschungskontext und Aufsetzen des begleitenden Beispiels ohne Formalisierung
        \item TIME: ~2
        \item TAKES: 5-6
    \end{itemize}
}

\begin{frame}{Motivation}
    How should a judge change their worldview when presented with new information?
    \note{
        \begin{itemize}
            \item Beispiel Frage zur Motivation, wird unser Beispiel sein. Unser Agent ist ein Richter in einem Mordprozess, wir haben zwei Verdächtige und diverse Beweise und Indizien, wie bilden wir uns ein Bild über die Welt?
            \item Iterierte Information: Immer wieder neue Informationen nacheinander mit der wir unseren GLauben immer wieder anpassen müssen.
            \item Nicht perfekte Information: Kein gesichertes Wissen sondern Zeugenaussagen/Indizien, können falsch sein, können unserem aktuellem Weltbild widersprechen
            \item Keine Frage mit richtig/falsch Antwort. Soll neuer Information immer geglaubt werden? Wie sollen Konflikte gelöst werden?
        \end{itemize}
    }
\end{frame}

\begin{frame}{Research Context}
    \begin{itemize}
        \item Philosophy and artificial intelligence \cite{Ferme2011}
        \item Belief change \cite{Darwiche1997}
              \begin{itemize}
                  \item Nonmonotonic logic
                  \item Probabilistic reasoning
                  \item Belief revision
              \end{itemize}
        \item One-step vs. iterated belief revision
    \end{itemize}

    \note{
        \begin{itemize}
            \item Einordnung des Papiers in einen Kontext
            \item Belief change, Glaubens änderung, Veränderung der Weltansicht mit neuen, unsicheren informationen
            \item Agent kann mensch oder maschine sein, Philosophie oder KI
            \item Unterschiedliche Lösungsansätze, nichtmonotone logik bekannt aus lehrstuhl kursen (truth maintenance systems, default logic wie reiter), probabilistisches denken (auch in Lehrstuhl kursen),
            \item Belief revision: operator basiert. Operatoren die einen Agentenstatus in einen neuen überführen mit neuen infromationen: definieren und Eigenschaften diskutieren
            \item One step vs iterated: Je nachdem ob man eine neue information betrachtet oder eine Reihe von neuen informationen
            \item Booth and Meyer: iterated belief revision
        \end{itemize}
    }
\end{frame}

\begin{frame}{Different types of belief}
    \begin{itemize}
        \item Belief set \cite{Alchourron1985}
        \item Conditional beliefs \cite{Darwiche1997}
        \item Change of conditional beliefs \cite{Booth2011}
    \end{itemize}

    \note{
        \begin{itemize}
            \item Glaubensmengen sind was ein agent aktuell als wahr akzeptiert (AGM theory: Wie ist eine glaubensmenge definiert und wie kann sie sich mit neuen informationen ändern?)
            \item Konditionale annahmen sind was ein agent bereit ist zu akzeptieren mit neuen informationen? (Darwiche and Pearl: Was ist die "strategie" für zukünftige glaubensmengen änderungen? Wie ändern sich konditionale glauben?)
            \item Booth and Meyer: Änderung in konditionalen glauben: Wenn konditionale glauben die strategie der änderung ist, wie können wir unsere strategie ändern?
        \end{itemize}
    }
\end{frame}

\begin{frame}{Research question}
    \begin{itemize}
        \item "conditional belief" revision operators
        \item axiomatisation of a family of operators by defining properties
        \item discuss properties and define concrete example
    \end{itemize}

    \note{
        \begin{itemize}
            \item Forschungsfrage des papiers
            \item Wie kann man konditionalen Glauben ändern mithilfe von Revisionsoperatoren
            \item Definition von postulaten die alle möglichen operatoren einschränken und damit eine familie von operatoren definieren über die wir reden können
            \item Konkreten beispieloperator definieren der diese Postulate erfüllt
        \end{itemize}
    }
\end{frame}

\section{Formal Background}
\note{
    \begin{itemize}
        \item Vorher: Verstehen der Forschungsfrage, Einordnung des Papiers in einen Forschungskontext und Aufsetzen des begleitenden Beispiels
        \item Jetzt: Formalisierung des Beispiels, Etablieren was eine total preorder ist.
        \item Jetzt: Einführung von Formalisierungen zu Glaubensmengen/Annahmen und wie preorders dort benutzt werden
        \item Jetzt: Postulate um Änderungen dieser Glaubensmengen/Annahmen mit neuen Informationen einzuschränken
        \item TIME: ~7
        \item TAKES: 10-11
    \end{itemize}
}

\begin{frame}{Courtroom example \footnote{inspired by \cite{Booth2011} and \cite{Darwiche1997}}}
    \label{slide:example-setup}
    \begin{itemize}
        \item The agent is a judge in a murder trial, "John" and "Mary" are suspects, the victim might be an alien
        \item $\Sigma = \{ p, q, r\}$
              \begin{itemize}
                  \item p = "John is the murderer"
                  \item q = "Mary is the murderer"
                  \item r = "The victim is an alien"
              \end{itemize}
        \item  $Int({\Sigma}) = W = \{ 000, 001, 010, 011, 100, 101, 110, 111\}$
        \item $Mod(p) = \modelsOf{p} = \{ 100, 101, 110, 111\}$, $100 \in \modelsOf{p}$
        \item Lower case greek letters used for propositional sentences $\alpha$
    \end{itemize}

    \note{
        \footnotesize
        \begin{itemize}
            \item Der Agent ist ein Richter in einem Mordprozess vor Gericht, John und Marry sind verdächtige und das Opfer könnte ein alien sein
            \item Beispiel inspiriert aus der Literatur, "übersetzt" in Syntax die evtl bekannt ist
            \item Logische Sprache, Aussagenlogik, Aussagenlogische Signatur mit Aussagenvariablen p, q, r
            \item "Mögliche Welt" als aussagenlogische Interpretationen/Belegungen, damit 8 Alternativen hier
            \item Kodierung der Welt als Tupel von Wahrheitswerten (0 = false, 1 = true)
            \item Beispiele für 1-2 Welten durchgehen: 010 = Mary ist Mörderin, Opfer ist kein Alien, 111 = John und Mary sind die Mörder, haben ein Alien getötet
            \item models of syntax, modelle von p
            \item Aussagenlogische formeln als kleine griechische buchstaben
        \end{itemize}
    }
\end{frame}

\begin{frame}{Belief Sets}
    \begin{itemize}
        \item Set of propositions the agent accepts as true at any point in time \cite{Ferme2011}
        \item Deductively closed
        \item Possible for example: $Cn(\{ p \vee q, \neg(p \wedge q), \neg r \})$
    \end{itemize}

    \note{
        \begin{itemize}
            \item Glaubensmengen als aktuell als wahr akzeptierte Sätze eines Agenten
            \item deduktiv abgeschlossen, d.h. alle formeln und logische folgerungen aus ihnen
            \item Betrachtungsgegenstand von AGM-Theorie \cite{Alchourron1985}, wie kann man sie ändern mit neuen informationen
            \item Zusätzlich bei menschlichem Schließen: Für wie wahrscheinlich halte ich etwas, wann wäre ich bereit es zu akzeptieren? -> Konditionale annahmen/glauben
        \end{itemize}
    }
\end{frame}

\begin{frame}{Epistemic States}
    \begin{itemize}
        \item abstract entity $\mathbb{E}$ that contain all information an agent need for their reasoning \cite{Darwiche1997}
        \item strategy for reasoning can be modeled as tpo $\leq_{\mathbb{E}}$ over worlds
        \item belief sets $B(\mathbb{E})$ can be extracted from epistemic states
              \begin{itemize}
                  \item set of minimal worlds in which $\alpha$ is true: $min(\alpha, \leq_{\mathbb{E}})$, set of most plausible worlds:  $min(\top, \leq_{\mathbb{E}})$
                  \item Set of most plausible worlds $min(\top, \leq_{\mathbb{E}})$
                  \item All sentences true in those worlds: $Th(min(\top, \leq_{\mathbb{E}}))$
              \end{itemize}
    \end{itemize}

    \note{
        \begin{itemize}
            \item  Strategie des Agenten wie er seinen Glauben ändert, damit auch die konditionalen annahmen.
            \item Darwiche and Pearl argumentieren das Belief sets nicht ausreichend sind, schlagen epistemische staten vor, abstrakt und enthalten glaubensmenge und strategie.
            \item Strategie kann in form einer tpo über die welten definiert werden
        \end{itemize}
    }
\end{frame}

\begin{frame}{Total preorders}
    \begin{itemize}
        \item Common tool to handle preference orderings over propositional worlds \cite{Booth2011}
        \item binary relation $\leq$, total, reflexive, transitive
        \item  $<$ strict, $\sim$ symmetric closure
    \end{itemize}
    \note{
        \begin{itemize}
            \item übliches werkzeug um präferenzen von möglichen welten zu modellieren
            \item total (for all $x, y \in W$ either $x \leq y$ or $y \leq x$)
            \item reflexive ($x \leq x$ for all $x \in W$)
            \item transitive (if $x \leq y$ and $y \leq z$ then $x \leq z$)
            \item  $<$ strict part of $\leq$
            \item $\sim$ (tilde) symmetrische relation auf of $\leq$ (i.e. $x \sim y$ iff $x \leq y$ and $y \leq x$)
        \end{itemize}
    }
\end{frame}

\begin{frame}{Preorder example}
    \begin{itemize}
        \item Judge beliefs
              \begin{itemize}
                  \item Murderer probably acted alone but possible that they conspired
                  \item Unlikely, but not impossible, for the victim to be an alien
              \end{itemize}
        \item $\leq$ over $W$: $010 \sim 100 < 000 \sim 110 < 011 \sim 101 < 001 \sim 111$
    \end{itemize}
    \note{
        \begin{itemize}
            \item Beispiel TPO anhand des richters der denkt einer der angeklagten ist wahrscheinlich der mörderer und hat alleine gehandelt. sehr unwahrscheinliche aber möglich, opfer ist ein alien.
            \item ein Mörderer und kein Alien gleich plausibel, dann kein/beide, dann jeweils mit Alien
        \end{itemize}
    }
\end{frame}

\begin{frame}{Preorder visualisation}
    \label{slide:vis-ranks}
    \begin{itemize}
        \item $[[x]]_{\sim} = \{y \mid y \sim x\}$
        \item $[[x]] \leq [[y]]$ iff $x \leq y$
    \end{itemize}

    \begin{table}[H]
        \centering
        \begin{tabular}{llll}
            $R_{1}$                                          & $R_{2}$                                         & $R_{3}$                                         & $R_{4}$ \\ \hline
            \multicolumn{1}{|l|}{\begin{tabular}[c]{@{}l@{}}$010$\\ $100$\end{tabular}} & \multicolumn{1}{l|}{\begin{tabular}[c]{@{}l@{}}$000$\\ $110$\end{tabular}} & \multicolumn{1}{l|}{\begin{tabular}[c]{@{}l@{}}$011$\\ $101$\end{tabular}} &
            \multicolumn{1}{l|}{\begin{tabular}[c]{@{}l@{}}$001$\\ $111$\end{tabular}}                                                                                                                \\ \hline
        \end{tabular}
        \caption{Visualizing a tpo as a linearly ordered set of ranks, as done in \cite{booth2006bad}}.
        \label{tab:visualising-a-tpo-example}
    \end{table}

    \note{
        \begin{itemize}
            \item TPOs können visualisiert werden als geordnete Liste von Rängen auf denen die Welten sind
            \item Die Ränge enthalten jeweils Welten mit gleicher Plausibilität
            \item Mathematisch ausgedrückt sind sie die Äquivalenzklassen zur Symmetrie (teilt die Welten in disjunkte Teilmengen ein)
        \end{itemize}
    }
\end{frame}

\begin{frame}{Belief Set Revision Postulates\footnote{AGM postulates \cite{Alchourron1985}, reformulated for epistemic states by Darwiche and Pearl \cite{Darwiche1997}}}
    \label{slide:e1-8}
    \begin{enumerate}[wide=0pt, widest=99,leftmargin=\parindent,label = ($\mathbb{E}\!*\!\arabic*$)]
        \item\label{E1} $\qquad B(\mathbb{E}\ast\alpha) = Cn(B(\mathbb{E}\ast\alpha))$
        \item\label{E2} $\qquad \alpha \in B(\mathbb{E}\ast\alpha)$
        \item\label{E3} $\qquad B(\mathbb{E}\ast\alpha)  \subseteq B(\mathbb{E})+\alpha$
        \item\label{E4} $\qquad \textrm{If } \neg \alpha \notin B(\mathbb{E}) \textrm{ then } B(\mathbb{E}) + \alpha \subseteq B(\mathbb{E} \ast \alpha)$
        \item\label{E5} $\qquad \textrm{If } \mathbb{E} = \mathbb{F} \textrm{ and } \alpha \equiv \beta \textrm{ then } B(\mathbb{E} \ast \alpha) = B(\mathbb{F} \ast \beta)$
        \item\label{E6} $\qquad \bot \in B(\mathbb{E} \ast \alpha) \textrm{ iff } \models \neg \alpha$
        \item\label{E7} $\qquad B(\mathbb{E} \ast (\alpha \wedge \beta)) \subseteq B(\mathbb{E} \ast \alpha) + \beta$
        \item\label{E8} $\qquad \textrm{If } \neg \beta \notin B(\mathbb{E} \ast \alpha) \textrm{ then } B(\mathbb{E} \ast \alpha) + \beta \subseteq B(\mathbb{E} \ast (\alpha \wedge \beta))$
    \end{enumerate}

    \note{
        \begin{itemize}
            \tiny
            \item AGM postulate für belief set änderung nach dem prinzip des minimal change, neu formuliert mit der epistemischen stati sicht
            \item $\mathbb{E}$ als epistemischer status, $B(\mathbb{E})$ belief set des status, $B(\mathbb{E}) + \alpha$ ist die expansion $B(\mathbb{E})$ mit $\alpha$ ($\alpha$ hinzufügen. $CN(B(\mathbb{E}) \cup \alpha)$), $\ast$ being a belief change operator on epistemic states
            \item 1: "Closure": deduktiv abgeschlossen, ist ein belief set
            \item 2: "Success": neue information ist immer teil des belief sets nach revision. Das muss nicht so sein, für BM tpo revision operatoren die wir hier betrachten gilt dies z.b. nicht. Wird "non-priotized revision" genannt
            \item 3 und 4: "Inclusion" und "Vacuity": zusammen, wenn neues belief ist konsistent mit dem vorherigen belief set dann gibt es keinen grund beliefs zu entfernen. das neue belief set wird vorheriges belief set + alpha + was aus alpha und belief set folgt sein, minimal change wenn alpha konsistent mit belief set
            \item 5: Syntax irrelevanz für information mit der revision betrieben wird. Epistemische zustände müssen gleich sein hier, im AGM original muss nur das belief set der zustände gleich sein. Also ist E5 schwächer als das Original von AGM.
            \item 6: Das belief set enthält elementaren widerspruch NUR nach revision mit inkonsistenten informationen, konsistenz "at all costs" (alpha wärend not alpha eine tautologie ist). BM bestehen auf konsistente Eingaben damit sie immer ein belief set aus dem epistemischen status ableiten können. Daher nehmen sie E6 nicht an und nur E1-5, E7-8 als "DP-AGM"
            \item 7 und 8: Wenn ein belief set nach revision mit alpha konsistent ist mit beta dann kann kann man das belief set von alpha und beta durch expansion mit beta erreichen (revision nicht nötig), wieder minimal change
        \end{itemize}
    }
\end{frame}

\begin{frame}{Conditional Belief Revision Postulates\footnote{by Darwiche and Pearl \cite{Darwiche1997}}}
    \label{slide:cr1-4}
    \begin{enumerate}[wide=0pt, widest=99,leftmargin=\parindent,label = (CR$\arabic*$)]
        \item\label{CR1} $\qquad \textrm{If } v\in \modelsOf{\alpha}, w \in \modelsOf{\alpha} \textrm{ then } v \leq_{\mathbb{E}} w \textrm{ iff } v \leq_{\mathbb{E\ast\alpha}} w$
        \item\label{CR2} $\qquad \textrm{If } v\in \modelsOf{\neg\alpha}, w \in \modelsOf{\neg\alpha} \textrm{ then } v \leq_{\mathbb{E}} w \textrm{ iff } v \leq_{\mathbb{E\ast\alpha}} w$
        \item\label{CR3} $\qquad \textrm{If } v\in \modelsOf{\alpha}, w \in \modelsOf{\neg\alpha} \textrm{ then } v <_{\mathbb{E}} w \textrm{ only if } v <_{\mathbb{E\ast\alpha}} w$
        \item\label{CR4} $\qquad \textrm{If } v\in \modelsOf{\alpha}, w \in \modelsOf{\neg\alpha} \textrm{ then } v \leq_{\mathbb{E}} w \textrm{ only if } v \leq_{\mathbb{E\ast\alpha}} w$
    \end{enumerate}
    \note{
        \begin{itemize}
            \item Zusätzliche postulate von DP um die änderung von konditionalen annahmen einzuschränken
            \item Semantische version aka wie sich die Ordnung der Welten ändert mit v,w in W
            \item CR1/2: Relative Ordnung der welten die alpha modelle/nicht alpha modelle sind bleibt gleich bei revision mit alpha (minimal change)
            \item CR3 wenn eine welt die ein alpha model ist strikt bevorzugt wurde vor einem nicht model dann muss das auch nach einer revision mit alpha gelten
            \item CR4 wie CR3 nur für schwache ordnung
        \end{itemize}
    }
\end{frame}

\section{Additional metadata for tpo revision}
\note{
    \begin{itemize}
        \item Vorher: Formalisierung des Beispiels/Glaubensmengen/Epistemischen Zuständen, Etablieren was eine total preorder ist, Postulate für Revision von Glaubensmengen und Annahmen
        \item Jetzt: BM wollen revision von tpos besprechen und führen dafür neue Metadaten und Strukturen im epistemischen Zustand ein, wir stellen diese nun vor bevor wir im nächsten Schritt die Änderung selbst besprechen
        \item Jetzt: Einführung einer intuitiven Visualisierung der neuen Struktur die wir im weiteren Verlauf nutzen
        \item TIME: ~17
        \item TAKES: 9-10
    \end{itemize}
}

\begin{frame}{Enriching Epistemic States}
    \begin{itemize}
        \item Additional structure $W^{\pm} = \{x^{\epsilon} \mid x \in W \textrm{ and } \epsilon \in \{+, -\}\}$
        \item Interval representing a world $(w^{+}, w^{-})$
        \item "A bad day surfing is better than a good day working" \cite{booth2006bad}
        \item Worlds are either supported by evidence or not $w \in \modelsOf{\alpha}$ / $w \in \modelsOf{\neg\alpha}$
    \end{itemize}
    \note{
        \begin{itemize}
            \item zusätzliche struktur W+- um unterschiedliche plausibilität von welten in W nach neuer information zu modelieren
            \item "A bad day surfing is better than a good day working" als idee das eine Welt sogar unter evidenz die sie nicht supported mehr plausibel bleiben kann, e.g. aliens tendenziell nicht so einfach zu glauben
        \end{itemize}
    }
\end{frame}

\begin{frame}{$\leq$-faithful tpo}
    \begin{itemize}
        \item original tpo $\leq$ was an order over $W$
        \item $\preceq$ over new $W^{\pm}$
    \end{itemize}

    \note{
        \begin{itemize}
            \item zusätzlich zu der neuen Repräsentation von Welten eine neue Ordnung über diese Intervalle
        \end{itemize}
    }
\end{frame}

\begin{frame}{$\leq$-faithful tpo - definition}
    \label{slide:faithful-tpo}
    \begin{enumerate}[wide=0pt, widest=99,leftmargin=\parindent,label = ($\preceq\arabic*$)]
        \item\label{PREQ1} $\qquad \preceq \textrm{ is a tpo over } W^{\pm}$
        \item\label{PREQ2} $\qquad x^{+} \preceq y^{+} \textrm{ iff } x \leq y$
        \item\label{PREQ3} $\qquad x^{-} \preceq y^{-} \textrm{ iff } x \leq y$
        \item\label{PREQ4} $\qquad x^{+} \prec x^{-}$
    \end{enumerate}

    \begin{definition}[$\leq$-faithful tpo over $W^{\pm}$ \cite{Booth2011}]
        \label{definition:faithful-tpo}Let $\preceq \subseteq W^{\pm} \times W^{\pm}$. If $\preceq$ satisfies \ref{PREQ1}-\ref{PREQ4}, we say $\preceq$ is a $\leq$-faithful tpo (over $W^{\pm}$).
    \end{definition}
    \note{
        \begin{itemize}
            \item 1. basic, preceq ist eine tpo über die neue intervall Struktur
            \item 2/3 faithful zur orginalen tpo, positive representation precqe iff es auch so in der originalen tpo ist
            \item das bedeutet die originale tpo kann von der faithful tpo rekonstruiert werden (einfach nur auf + oder - einschränken und alte tpo ablesen)
            \item daher muss man nur neue tpo und W+- speichern im epistemischen Zustand
            \item 4 bedeutet es muss einen unterschied geben zwischen den representation, interval kann nicht länge 0 haben
        \end{itemize}
    }
\end{frame}

\begin{frame}[fragile]{$\leq$-faithful tpo visualisation}
    \begin{figure}[H]
        \centering
        \begin{tikzpicture}[scale=1]
            \begin{axis}[
                    yticklabels={,,$x_{4}$,$x_{3}$,$x_{2}$,$x_{1}$},
                    xticklabels={,,},
                    ytick style={draw=none},
                    xtick style={draw=none},
                    axis line style={draw=none}
                ]
                \addplot[
                    scatter,
                    scatter src=explicit symbolic,
                    mark size=3,
                    scatter/classes={
                            empty={mark=*, fill=white},
                            filled={mark=*, fill=black}
                        },
                    nodes near coords*={\Label},
                    visualization depends on={value \thisrow{label} \as \Label}
                ]
                table [meta=class] {
                        x y class label

                        3 1 empty $x_{4}^{+}$
                        6 1 empty $x_{4}^{-}$

                        2 2 empty $x_{3}^{+}$
                        5 2 empty $x_{3}^{-}$

                        1 3 empty $x_{2}^{+}$
                        3 3 empty $x_{2}^{-}$

                        1 4 empty $x_{1}^{+}$
                        3 4 empty $x_{1}^{-}$
                    };
            \end{axis}
        \end{tikzpicture}
        \caption{Representation of $\preceq$ over $W^{\pm}$ using intervals}
        \label{fig:example-visualisation-scatterplot}
    \end{figure}

    \note{
        \begin{itemize}
            \item beispielvisualisierung, x1+- hier zusätzlich dran geschrieben zur einführung, wird nachher weggelassen
            \item zeilen sind die welten, spalten kann man sich als Ränge in der TPO vorstellen
            \item je weiter links desto plausibler
            \item Interval zwischen + und - ergibt sich aus proposition 4 vorher, muss nicht gleich sein aber größer als 0
        \end{itemize}
    }
\end{frame}

\begin{frame}[fragile]{Courtroom example: $\leq$-faithful tpo}
    \begin{figure}[H]
        \centering
        \begin{tikzpicture}[scale=1]
            \begin{axis}[
                    xticklabels={},
                    yticklabels={},
                    extra y ticks={1, 2, 3, 4, 5, 6, 7, 8},
                    extra y tick labels={$111$, $001$, $101$, $011$, $110$, $000$, $100$, $010$},
                    ytick style={draw=none},
                    xtick style={draw=none},
                    axis line style={draw=none}
                ]
                \addplot[
                    scatter,
                    scatter src=explicit symbolic,
                    mark size=3,
                    scatter/classes={
                            empty={mark=*, fill=white},
                            filled={mark=*, fill=black}
                        },
                    nodes near coords*={\Label},
                    visualization depends on={value \thisrow{label} \as \Label}
                ]
                table [meta=class] {
                        x y class label

                        1 8 empty \empty
                        3 8 empty

                        1 7 empty
                        3 7 empty

                        2 6 empty
                        4 6 empty

                        2 5 empty
                        4 5 empty

                        5 4 empty
                        7 4 empty

                        5 3 empty
                        7 3 empty

                        6 2 empty
                        8 2 empty

                        6 1 empty
                        8 1 empty
                    };
            \end{axis}
        \end{tikzpicture}
        \caption{Representation of $\preceq$ over $W^{\pm}$ for the courtroom example}
        \label{fig:example-tpo-initial}
    \end{figure}

    \note{
        \begin{itemize}
            \item Visualisierung für unser Beispiel
            \item Erinnerung an die vier Äquivalenzklassen von vorher: 1 Mörderer, kein/zwei Mörderer und beides jeweils mit Alien
            \item Anmerken: Überlappung zwischen den positiven/negativen Repräsentationen im nicht Alien Fall und Abstand zwischen Alien und nicht Alien, gehen wir nachher noch drauf ein aber hätte man vorher nur mit W nicht modellieren können, Unterschiedlich unwahrscheinliche Welten
        \end{itemize}
    }
\end{frame}

\section{Booth and Meyer tpo-revision operators}
\note{
    \begin{itemize}
        \item Vorher: Neue Metadaten/Strukturen im epistemischen Zustand und Visualisierung
        \item Jetzt: Nutzen dieser neuen Strukturen für tpo revision, Postulates für BM revision operatoren und Definition der Familie von BM revisons operatoren
        \item Jetzt: Abstrakte tpo revision am Beispiel
        \item TIME: ~26
        \item TAKES: 19
    \end{itemize}
}

\begin{frame}{BM tpo-revision operator}
    \label{slide:bm-revision-operator}
    \begin{definition}[Revision operator $\ast_{\preceq}$ for $\leq$ generated by $\preceq$ \cite{Booth2011}]
        \label{definition:revision-operator}
        For each $\leq$-faithful tpo $\preceq$ over $W^{\pm}$, refer to $\ast_{\preceq}$ as the revision operator for $\leq$ generated by $\preceq$ defined by:

        Set for any $\alpha \in L$ and $x \in W$:
        \begin{equation*}
            r_{\alpha}(x) = \left\{
            \begin{array}{ll}
                x^{+} \textrm{ if } x \in \modelsOf{\alpha} \\
                x^{-} \textrm{ if } x \in \modelsOf{\neg\alpha}
            \end{array}
            \right.
        \end{equation*}

        The revised tpo $\leq_{\alpha}^{\ast}$ is defined by setting, for each $x, y \in W$,

        \begin{equation*}
            x \leq_{\alpha}^{\ast} y \textrm{ iff } r_{\alpha}(x) \preceq r_{\alpha}(y)
        \end{equation*}
    \end{definition}

    \note{
        \begin{itemize}
            \item Definition eines BM revisions operators
            \item Funktion * von tpo und satz $\alpha$ aus L zu neuer tpo auf W, basierend auf unserer extra struktur auf W+-
            \item Syntax: neue tpo als tpo subscript alpha, generiert mit superskript *
            \item Prinzip einfach: Für jeden satz alpha und welt x, ASSOZIIERE x mit ihrer positive representation x+ in w+ wenn x ein model von alpha ist und auf ihre negative representation x- wenn sie kein model von alpha ist
            \item Die neue TPO ergibt sich dann aus der Ordnung auf den jeweiligen positiven/negativen representationen
            \item Damit operatorbasierte belief revision, konditionale glauben (strategie) in form der tpo und glaubensset weil man es aus der tpo extrahieren kann
        \end{itemize}
    }
\end{frame}

\begin{frame}[fragile]{Courtroom example: Revision Visualised}

    \begin{figure}[H]
        \centering
        \begin{tikzpicture}[scale=1]
            \begin{axis}[
                    xticklabels={},
                    yticklabels={},
                    extra y ticks={1, 2, 3, 4, 5, 6, 7, 8},
                    extra y tick labels={$111$, $001$, $101$, $011$, $110$, $000$, $100$, $010$},
                    ytick style={draw=none},
                    xtick style={draw=none},
                    axis line style={draw=none}
                ]
                \addplot[
                    scatter,
                    scatter src=explicit symbolic,
                    mark size=3,
                    scatter/classes={
                            empty={mark=*, fill=white},
                            filled={mark=*, fill=black}
                        },
                    nodes near coords*={\Label},
                    visualization depends on={value \thisrow{label} \as \Label}
                ]
                table [meta=class] {
                        x y class label

                        1 8 empty \empty
                        3 8 filled

                        1 7 filled
                        3 7 empty

                        2 6 empty
                        4 6 filled

                        2 5 filled
                        4 5 empty

                        5 4 empty
                        7 4 filled

                        5 3 filled
                        7 3 empty

                        6 2 empty
                        8 2 filled

                        6 1 filled
                        8 1 empty
                    };
            \end{axis}
        \end{tikzpicture}
        \caption{Associating positive and negative representations of worlds after receiving evidence $\alpha=p$}
        \label{fig:example-tpo-revised}
    \end{figure}

    \note{
        \footnotesize
        \begin{itemize}
            \item Hier beispiel von vorher mit alpha gleich p "John ist der Mörderer"
            \item r subskript alpha von x ist hier jeweils als schwarzer Punkt markiert
            \item Für Welten die Modelle von alpha sind (wie 100 oder 110) wird ihre positive representation gewählt
            \item Für Welten die keine Modelle sind (wie 010 "Mary ist die Mörderin") die negative representation
            \item Anmerkung: Welten die keine Modelle sind können trotzdem noch plausibler sein, siehe 010 und 101.
            \item Es macht intuitiv Sinn: Ein Richter würde eher annehmen, dass Mary die Mörderin ist obwohl es Indizien für John gibt als zu akzeptieren, dass das Opfer ein Alien ist
            \item Das wird noch kommen als "non-priotized revision"
        \end{itemize}
    }
\end{frame}

\begin{frame}{Courtroom example: Revision 1}

    \begin{itemize}
        \item for $010, 110 \in W$, $010 < 110$ before, revise by $\alpha = p$
    \end{itemize}

    \begin{equation*}
        010 \in \modelsOf{\neg\alpha} \textrm{ : } r_{\alpha}(010) = 010^{-}
    \end{equation*}
    \begin{equation*}
        110 \in \modelsOf{\alpha} \textrm{ : } r_{\alpha}(110) = 110^{+}
    \end{equation*}

    \begin{itemize}
        \item $110^{+} \prec 010^{-}$ is true, set $110 <_{\alpha}^{\ast} 010$
        \item new tpo $\leq_{\alpha}^{\ast}$ is: $100 <_{\alpha}^{\ast} 110 <_{\alpha}^{\ast} 010 <_{\alpha}^{\ast} 000 <_{\alpha}^{\ast} 101 <_{\alpha}^{\ast} 111 <_{\alpha}^{\ast} 011 <_{\alpha}^{\ast} 001$
    \end{itemize}

    \note{
        \footnotesize
        \begin{itemize}
            \item Hier für 010 "Nur Mary ist die Mörderin" und 110 "Mary und John sind die Mörder"
            \item Vorher: Zwei Mörderer sind unwahrscheinlich, daher ist nur Mary ist die Mörderin plausibler
            \item Evidenz für p "Nur John ist der Mörder"
            \item 010 "Nur Mary ist die Mörderin" ist kein Model also wird ihr die negative Representation 010- zugewiesen
            \item 110 "Mary und John sind die Mörder" ist ein Model also wird ihr die positive Representation 110+ zugewiesen
            \item zurückgehen zur grafik und zeigen
            \item Da die positive Representation 110+ für "Mary und John sind die Mörderer" plausibler ist als die negative Representation von 010- Nur  Mary ist die Mörderin ist nach der Evidenz 110 (beide sind Mörderer) plausibler
            \item neue TPO nachdem man diesen Prozess für alle Welten macht
        \end{itemize}
    }
\end{frame}

\begin{frame}{Courtroom example: Revision 2}
    \begin{itemize}
        \item new tpo $\leq_{\alpha}^{\ast}$ is: $100 <_{\alpha}^{\ast} 110 <_{\alpha}^{\ast} 010 <_{\alpha}^{\ast} 000 <_{\alpha}^{\ast} 101 <_{\alpha}^{\ast} 111 <_{\alpha}^{\ast} 011 <_{\alpha}^{\ast} 001$
        \item $min(\top, \leq_{\alpha}^{\ast} ) = \{100\}$: "John is the murderer and the victim is not an alien".
        \item $\leq_{\alpha}^{\ast}$ as representation of the conditional beliefs
              \begin{itemize}
                  \item before $010 < 110$: "Both suspects being the murderer is less plausible than only Mary being the murderer"
                  \item now $110 <_{\alpha}^{\ast} 010$: "Only Mary being the murderer less plausible than both conspiring".
              \end{itemize}
    \end{itemize}

    \note{
        \begin{itemize}
            \item Das belief set hat sich geändert: Das belief set der neuen tpo ist nun alle sätze die wahr sind in der plausibelsten welt: "John ist der mörderer und das opfer ist kein alien"
            \item Macht intuitiv Sinn
            \item Auch die conditional beliefs des agenten haben sich geändert mit der TPO: Vorher beide Mörderer nicht so plausibel wie nur Mary, jetzt beide Mörderer plausibler als nur Mary (weil wir ja evidenz haben das zumindest John ein Mörderer ist)
            \item Macht intuitiv auch Sinn
        \end{itemize}
    }
\end{frame}

\begin{frame}{Properties of BM Revision Operators: Basic properties}
    \label{slide:ast1-7}
    \begin{enumerate}[wide=0pt, widest=99,leftmargin=\parindent,label = ($\ast\arabic*$)]
        \item\label{AST1} $\qquad \leq_{\alpha}^{\ast} \textrm{ is a tpo over } W$
        \item\label{AST2} $\qquad\alpha \equiv \gamma \textrm{ implies } \leq_{\alpha}^{\ast}=\leq_{\gamma}^{\ast}$
    \end{enumerate}

    \note{
        \begin{itemize}
            \item Was sind Eigenschaften von BM revisions operatoren? Diese Eigenschaften sind gruppiert, ich werde sie durchgehen und am Ende werden sie die Familie von operatoren kategorisieren/einschränken über die BM reden
            \item \ref{AST1} und \ref{AST2} sind basis Eigenschaften
            \item Eine revision einer tpo über W muss wieder eine tpo über W sein
            \item Für semantisch äquivalente Sätze muss der operator die gleiche neue tpo erzeugen, so genannte syntax irrelevance property
        \end{itemize}
    }
\end{frame}

\begin{frame}{Properties of BM Revision Operators: Common rules in iterated belief change}
    \begin{enumerate}[wide=0pt, widest=99,leftmargin=\parindent,label = ($\ast\arabic*$)]
        \setcounter{enumi}{2}
        \item\label{AST3} $\qquad \textrm{If } x, y \in \modelsOf{\alpha} \textrm{ then } x \leq_{\alpha}^{\ast} y \textrm{ iff } x \leq y$
        \item\label{AST4} $\qquad \textrm{If } x, y \in \modelsOf{\neg\alpha} \textrm{ then } x \leq_{\alpha}^{\ast} y \textrm{ iff } x \leq y$
        \item\label{AST5} $\qquad \textrm{If } x \in \modelsOf{\alpha}, y \in \modelsOf{\neg\alpha} \textrm{ and } x \leq y \textrm{ then } x <_{\alpha}^{\ast} y$
    \end{enumerate}

    \note{
        \footnotesize
        \begin{itemize}
            \item 3-5 regeln für iterated belief change mit einem input alpha
            \item 3 und 4: die relative Ordnung für Welten die entweder beide Modelle für die Eingabe sind oder nicht muss gleich bleiben (schon gegeben in conditional belief revision postulates CR1/CR2 von Darwiche and pearl)
            \item 5: wenn eine welt x mindestens so plausibel ist wie eine Welt y soll x nach einer Eingabe für die x ein model ist und y nicht strikt plausibler sein (ähnlich wie CR3/4 von DP aber strenger)
            \item BM vergleichen sie mit den AGM postulaten, AGM postulate für revision von belief sets, postulate 1-5 für revision von tpos
            \item (charakteristisch für "admissible revision operators" die eingaben nicht ignorieren, selbst wenn es nicht zu Änderungen am belief set führt (siehe 5))
            \item (im gegensatz zu e.g. natural revision von boutilier die Änderungen in conditional beliefs minimiert)
        \end{itemize}
    }
\end{frame}

\begin{frame}{Properties of BM Revision Operators: Supplementary rationality properties }
    \begin{enumerate}[wide=0pt, widest=99,leftmargin=\parindent,label = ($\ast\arabic*$)]
        \setcounter{enumi}{5}
        \item\label{AST6} $\qquad \textrm{If } x \in \modelsOf{\alpha}, y \in \modelsOf{\neg\alpha} \textrm{ and } y \leq_{\alpha}^{\ast} x \textrm{ then } y \leq_{\gamma}^{\ast} x$
        \item\label{AST7} $\qquad \textrm{If } x \in \modelsOf{\alpha}, y \in \modelsOf{\neg\alpha} \textrm{ and } y <_{\alpha}^{\ast} x \textrm{ then } y <_{\gamma}^{\ast} x$
    \end{enumerate}

    \note{
        \begin{itemize}
            \item vorher nur eine eingabe, diese zusätzlichen postulate um revision mit unterschiedlichen eingaben stimmig zu halten
            \item 6: wenn nach einer revision mit alpha die welt y mindestens so plausibel ist wie x obwohl x ein model für alpha ist und y nicht, dann ist y für jede eingabe gamme mindestens so plausibel wie x
            \item 7: gleich, nur für die strikt plausiblere version
        \end{itemize}
    }
\end{frame}

\begin{frame}{Family of BM Revision Operators}
    \label{slide:theorem1}
    \begin{theorem}
        \label{theorem:revision-operator}Let $\ast$ be any revision operator for $\leq$. Then $\ast$ is generated from some $\leq$-faithful tpo $\preceq$ over $W^{\pm}$ iff $\ast$ satisfies \ref{AST1}-\ref{AST7}. \cite{Booth2011}
    \end{theorem}

    \note{
        \begin{itemize}
            \item theorem 1 als definition der funktionen/revisionsoperatoren für eine tpo die BM diskutieren/vorschlagen
            \item erinnern an faithful tpo, W+-, 1-7
        \end{itemize}
    }
\end{frame}

\begin{frame}[fragile]{Non-priotized revision}
    \begin{figure}[H]
        \centering
        \begin{tikzpicture}[scale=1]
            \begin{axis}[
                    xticklabels={},
                    yticklabels={},
                    extra y ticks={1, 2, 3, 4, 5, 6, 7, 8},
                    extra y tick labels={$111$, $001$, $101$, $011$, $110$, $000$, $100$, $010$},
                    ytick style={draw=none},
                    xtick style={draw=none},
                    axis line style={draw=none}
                ]
                \addplot[
                    scatter,
                    scatter src=explicit symbolic,
                    mark size=3,
                    scatter/classes={
                            empty={mark=*, fill=white},
                            filled={mark=*, fill=black}
                        },
                    nodes near coords*={\Label},
                    visualization depends on={value \thisrow{label} \as \Label}
                ]
                table [meta=class] {
                        x y class label

                        1 8 empty \empty
                        3 8 filled

                        1 7 empty
                        3 7 filled

                        2 6 empty
                        4 6 filled

                        2 5 empty
                        4 5 filled

                        5 4 filled
                        7 4 empty

                        5 3 filled
                        7 3 empty

                        6 2 filled
                        8 2 empty

                        6 1 filled
                        8 1 empty
                    };
            \end{axis}
        \end{tikzpicture}
        \caption{Non-prioritised revision by $\alpha = r$}
        \label{fig:example-non-prio-revision}
    \end{figure}

    \note{
        \footnotesize
        \begin{itemize}
            \item Es gibt postulate die erfordern, dass neuen Informationen immer geglaubt wird und sie in das belief set aufgenommen werden, e.g. \ref{E2} als $\alpha \in B(\mathbb{E}\ast\alpha)$
            \item hier im Beispiel revision mit p "das Opfer ist ein Alien", modelle/nicht modelle jeweils mit positiver/negativer representation identifizieren (schwarzer punkt), neue tpo ablesen. hier trotz evidenz "das opfer ist ein alien", min worlds 010, 100 aka teil des belief sets ist "das opfer ist KEIN alien"
            \item BM revision operatoren sind "non priotized revision", d.h. neue informationen müssen nicht teil des belief sets werden
            \item aber auch erinnerung: admissible revision von BM erfordert (\ref{AST3} - \ref{AST5}), dass neue informationen nicht ignoriert werden, hier r verändert die conditional beliefs. das im gegensatz zu natural revision by boutilier welches minimal change für conditional beliefs fordert
            \item Wann glaubt der Richter denn an aliens? Welche zusätzlichen beweise braucht er und wie iterieren wir ein weiteres mal jetzt? I am glad you asked!
        \end{itemize}
    }
\end{frame}

\section{Iterating: \texorpdfstring{$\preceq$}{tpo}-revision}
\note{
    \begin{itemize}
        \item Vorher: BM revisions operatoren definition, Revision der tpo am Beispiel
        \item Jetzt: Revision der neuen Struktur $\preceq$, ansonsten Problem mehrmaliger Revision nur "hochgeschoben"
        \item Jetzt: Ein möglicher konkreter Beispieloperator + Visualisierung
        \item Jetzt: Wie die Revision von Annahmen funktioniert für mehr als einen Input
        \item TIME: ~45
        \item TAKES: 15
    \end{itemize}
}

\begin{frame}{A concrete operator \footnote{as shown in \cite{Booth2011}}: Setup}
    \begin{itemize}
        \item Function p mapping worlds to real numbers: $p: W^{\pm} \mapsto \mathds{R}$
        \item Interval representing a world $x$: $(p(x^{+}), p(x^{-}))$
        \item Distance between representations: $p(x^{-}) - p(x^{+}) = a > 0$
        \item Define a tpo from $p$: $x^{\epsilon} \preceq_{p} y^{\delta} \textrm{ iff } p(x^{\epsilon}) \leq p(y^{\delta})$
    \end{itemize}

    \note{
        \begin{itemize}
            \item explizit nur ein beispiel, nicht der einzige mögliche operator
            \item definiert über eine funktion p die welten aus w+- echten zahlen zuweist (je kleiner desto plausibler)
            \item damit wird ein interval für eine welt x aus W zu einem intervall von echten zahlen
            \item aufgrund der faithful tpo definition (slide 20) postulat 4 ist x+ < x-, daher distanz a des intervals > 0
            \item jetzt kann man mit p eine tpo definieren indem man welten in der tpo nach ihrer zahl in p anordnet
        \end{itemize}
    }
\end{frame}

\begin{frame}{A concrete operator: Iteration}
    \begin{itemize}
        \item Choose initial $p$ so that $\preceq_{p} = \preceq$
        \item Revise $p$ by $\alpha$ to $p \ast \alpha$, for every $x^{\epsilon} \in W^{\pm}$:
    \end{itemize}

    \begin{equation*}
        (p \ast \alpha)(x^{\epsilon}) = \left\{
        \begin{array}{ll}
            p(x^{\epsilon}) \textrm{ if } x \in \modelsOf{\alpha} \\
            p(x^{\epsilon}) + a \textrm{ if } x \in \modelsOf{\neg\alpha}
        \end{array}
        \right.
    \end{equation*}

    \begin{itemize}
        \item Define a revised tpo $\preceq_{p \ast \alpha}$ from $p \ast \alpha$: $x^{\epsilon} \preceq_{p \ast \alpha} y^{\delta} \textrm{ iff } (p \ast \alpha)(x^{\epsilon}) \leq (p \ast \alpha)(y^{\delta})$
    \end{itemize}

    \note{
        \begin{itemize}
            \item revision der übergeordneten tpo ist jetzt ein zwei schritt prozess:
            \item 1. schritt: mit der funktion p kann man eine tpo definieren, wähle p so das die damit definierte tpo = unserer übergeordneten tpo auf w+- ist
            \item 2. schritt: p mit alpha zu p alpha ändern indem man alle welten die ein model von alpha sind unverändert lässt und alle anderen um a "zurückschiebt" (x+ wird zur gleichen zahl wie x- da um a zurückgeschoben und x- zu x- + a)
            \item von der neuen funktion p alpha kann man jetzt eine neue übergeordnete tpo ablesen
        \end{itemize}
    }
\end{frame}

\begin{frame}{Courtroom Example: A concrete operator}
    \begin{itemize}
        \item $\preceq_{p \ast \alpha}$ for $\alpha = p$
        \item Choose initial $p$ so that $\preceq_{p} = \preceq$
              \begin{itemize}
                  \item $010 \textrm{ : } (p(010^{+}), p(010^{-})) = (0, a)$.
                  \item $100 \textrm{ : } (p(100^{+}), p(100^{-})) = (0, a)$.
              \end{itemize}
        \item Revise $p$ by $\alpha$ to $p \ast \alpha$
              \begin{itemize}
                  \item $010 \in \modelsOf{\neg\alpha} \textrm{ : } (p(010^{-}), p(010^{-}) + a) = (a, 2a)$
                  \item $100 \in \modelsOf{\alpha} \textrm{ : } (p(100^{+}), p(100^{-})) = (0, a)$
              \end{itemize}
    \end{itemize}

    \note{
        \begin{itemize}
            \item hier an unserem beispiel für alpha "John ist der mörderer"
            \item initial hat der richter keine präferenz welcher verdächtige der mörder ist, die welten 010 und 100 sind am plausibelsten (am weitesten links) und haben das gleiche interval. hier mit p zu 0 im positiven fall und a im negativen fall abgebildet.
            \item nachdem wir p zu p alpha verändern sind die intervalle:
            \item für das nicht model 010 vom START p(010+) + a = p(010-) zum ENDE p(010-) + a, also (a, 2a) "um a zurück geschoben"
            \item für das model 100 unverändert als (0, a)
            \item das kann man nun für alle welten machen und die neue übergeordnete tpo von p ableiten
        \end{itemize}
    }
\end{frame}

\begin{frame}[fragile]{Courtroom Example: Visualised}
    \begin{figure}[H]
        \centering
        \begin{tikzpicture}[scale=1]
            \begin{axis}[
                    xticklabels={},
                    yticklabels={},
                    extra y ticks={1, 2, 3, 4, 5, 6, 7, 8},
                    extra y tick labels={$111$, $001$, $101$, $011$, $110$, $000$, $100$, $010$},
                    ytick style={draw=none},
                    xtick style={draw=none},
                    axis line style={draw=none}
                ]
                \addplot[
                    scatter,
                    scatter src=explicit symbolic,
                    mark size=3,
                    scatter/classes={
                            empty={mark=*, fill=white},
                            old={mark=x, fill=white}
                        },
                    nodes near coords*={\Label},
                    visualization depends on={value \thisrow{label} \as \Label}
                ]
                table [meta=class] {
                        x y class label

                        1 8 old \empty

                        3 8 empty
                        5 8 empty

                        1 7 empty
                        3 7 empty

                        2 6 old

                        4 6 empty
                        6 6 empty

                        2 5 empty
                        4 5 empty

                        5 4 old

                        7 4 empty
                        9 4 empty

                        5 3 empty
                        7 3 empty

                        6 2 old

                        8 2 empty
                        10 2 empty

                        6 1 empty
                        8 1 empty
                    };
            \end{axis}
        \end{tikzpicture}
        \caption{$\preceq_{p \ast \alpha}$ for $\alpha = p$}
        \label{fig:example-preceq-revised}
    \end{figure}

    \note{
        \begin{itemize}
            \item hier visualisiert revision der übergeordneten tpo auf w+- mit alpha = p = John ist der mörderer
            \item intervalle der welten in der tpo p sternchen alpha dargestellt wie vorher
            \item alte positive repräsentationen vor der revision von nicht modellen markiert als x
            \item alle nicht modelle von p um a zurück geschoben, alle modelle bleiben gleich
            \item damit neue tpo ablesbar, minimale welt 100, neues belief set "John ist der mörderer" wie vorher und intuitiv
            \item was hat sich jetzt geändert? die konditionalen annahmen, was der agent bereit ist zu akzeptieren mit neuen beweisen
        \end{itemize}
    }
\end{frame}

\begin{frame}[fragile]{Courtroom Example: Conditional beliefs 1}
    \begin{figure}[H]
        \centering
        \begin{tikzpicture}[scale=1]
            \begin{axis}[
                    xticklabels={},
                    yticklabels={},
                    extra y ticks={1, 2, 3, 4, 5, 6, 7, 8},
                    extra y tick labels={$111$, $001$, $101$, $011$, $110$, $000$, $100$, $010$},
                    ytick style={draw=none},
                    xtick style={draw=none},
                    axis line style={draw=none}
                ]
                \addplot[
                    scatter,
                    scatter src=explicit symbolic,
                    mark size=3,
                    scatter/classes={
                            empty={mark=*, fill=white},
                            old={mark=x, fill=white}
                        },
                    nodes near coords*={\Label},
                    visualization depends on={value \thisrow{label} \as \Label}
                ]
                table [meta=class] {
                        x y class label

                        1 8 old \empty

                        3 8 empty
                        5 8 empty

                        1 7 old

                        3 7 empty
                        5 7 empty

                        2 6 old

                        4 6 empty
                        6 6 empty

                        2 5 old

                        4 5 empty
                        6 5 empty

                        5 4 empty
                        7 4 empty

                        5 3 empty
                        7 3 empty

                        6 2 empty
                        8 2 empty

                        6 1 empty
                        8 1 empty
                    };
            \end{axis}
        \end{tikzpicture}
        \caption{$\preceq_{p \ast \alpha}$ for $\alpha = r$}
        \label{fig:example-preceq-revised-alien}
    \end{figure}

    \note{
        \begin{itemize}
            \item hier, revision mit alpha = r = "Das opfer war ein alien"
            \item wie vorher besprochen, trotz der evidenz ist "das opfer ist ein alien" nicht teil der minimalen welten und damit nicht teil des belief sets (non priotized revision)
            \item aber die konditionalen annahmen haben sich jetzt verändert, der abstand zwischen den alien welten und nicht alien welten ist weg
            \item 1. tpo revision, 1. neue tpo ablesbar
        \end{itemize}
    }
\end{frame}

\begin{frame}[fragile]{Courtroom Example: Conditional beliefs 2}
    \begin{figure}[H]
        \centering
        \begin{tikzpicture}[scale=1]
            \begin{axis}[
                    xticklabels={},
                    yticklabels={},
                    extra y ticks={1, 2, 3, 4, 5, 6, 7, 8},
                    extra y tick labels={$111$, $001$, $101$, $011$, $110$, $000$, $100$, $010$},
                    ytick style={draw=none},
                    xtick style={draw=none},
                    axis line style={draw=none}
                ]
                \addplot[
                    scatter,
                    scatter src=explicit symbolic,
                    mark size=3,
                    scatter/classes={
                            empty={mark=*, fill=white},
                            filled={mark=*, fill=black},
                            old={mark=x, fill=white}
                        },
                    nodes near coords*={\Label},
                    visualization depends on={value \thisrow{label} \as \Label}
                ]
                table [meta=class] {
                        x y class label

                        1 8 old \empty

                        3 8 filled
                        5 8 empty

                        1 7 old

                        3 7 filled
                        5 7 empty

                        2 6 old

                        4 6 empty
                        6 6 filled

                        2 5 old

                        4 5 empty
                        6 5 filled

                        5 4 filled
                        7 4 empty

                        5 3 filled
                        7 3 empty

                        6 2 empty
                        8 2 filled

                        6 1 empty
                        8 1 filled
                    };
            \end{axis}
        \end{tikzpicture}
        \caption{$\preceq_{p \ast \alpha \ast \beta}$ for $\beta = (p \vee q) \wedge (\neg p \vee \neg q)$}
        \label{fig:example-preceq-revised-alien2}
    \end{figure}

    \note{
        \begin{itemize}
            \item jetzt revision mit beta "der mörder hat alleine gehandelt" nachdem wir vorher mit alpha "das opfer war ein alien" revision betrieben haben
            \item jetzt sind die welten mit dem opfer als alien und einem einzeltäter plausibler als die welten ohne alien aber ohne mörder/mit zwei mördern!
            \item hätten wir direkt mit beta revision betrieben wäre das nicht passiert, iterated belief revision am werk
            \item 2. tpo revision, 2. tpo
            \item allerdings hier: die welten mit einem einzeltäter und ohne alien (010 und 100) sind immernoch die minimalen welten und damit ist r "das opfer ist ein alien" nicht teil des belief sets
        \end{itemize}
    }
\end{frame}

\begin{frame}{}
    \centering \Huge
    Thank You
\end{frame}
\note{
    \begin{itemize}
        \item Vorher: Revision der neuen Struktur $\preceq$, ein möglicher konkreter Beispieloperator + Visualisierung
        \item Jetzt: Fragen? Diskussion
        \item Example setup: \pageref{slide:example-setup} - 1
        \item Visualisierung ranks: \pageref{slide:vis-ranks} -1
        \item E1-8: \pageref{slide:e1-8} -1
        \item CR1-4: \pageref{slide:cr1-4} -1
        \item faithful-tpo: \pageref{slide:faithful-tpo} -1
        \item bm-revision-operator: \pageref{slide:bm-revision-operator} -1
        \item ast1-7: \pageref{slide:ast1-7} -1
        \item theorem1: \pageref{slide:theorem1} -1
    \end{itemize}
}

\begin{frame}[allowframebreaks=0.9]{References}
    \typeout{}
    \bibliographystyle{apalike}
    \bibliography{references}
\end{frame}

\end{document}
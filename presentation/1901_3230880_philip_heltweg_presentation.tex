% TODO: explain contraction, revision, expansion
% ANSWER: Ich würdes es ganz kurz erläutern im Text. Im Vortrag vielleicht nicht umbedingt.
\documentclass[11pt]{beamer}
\usepackage[utf8]{inputenc}

\usefonttheme{professionalfonts}
\usetheme{Madrid}
\usecolortheme{orchid}


\AtBeginSection[]
{
  \begin{frame}
    \frametitle{Table of Contents}
    \tableofcontents[currentsection]
  \end{frame}
}

\begin{document}
\title{Of judges, aliens and total preorders}
\author[Heltweg, Philip]{Heltweg, Philip\\pheltweg@gmail.com}
\institute{University of Hagen}
\date{\today}

\frame{\titlepage}

\begin{frame}
    \frametitle{Table of Contents}
    \tableofcontents
\end{frame}

\section{Example}

\begin{frame}
    \frametitle{Sample frame title}
    This is a text in the first frame. This is a text in the first frame. This is a text in the first frame.
\end{frame}

\begin{frame}
\frametitle{Sample frame title}

In this slide, some important text will be
\alert{highlighted} because it's important.
Please, don't abuse it.

\begin{block}{Remark}
Sample text
\end{block}

\begin{alertblock}{Important theorem}
Sample text in red box
\end{alertblock}

\begin{examples}
Sample text in green box. The title of the block is ``Examples".
\end{examples}
\end{frame}

\section{More examples}

\begin{frame}
\frametitle{Two-column slide}

\begin{columns}

\column{0.5\textwidth}
This is a text in first column.
$$E=mc^2$$
\begin{itemize}
\item First item
\item Second item
\end{itemize}

\column{0.5\textwidth}
This text will be in the second column
and on a second tought this is a nice looking
layout in some cases.
\end{columns}
\end{frame}

\end{document}

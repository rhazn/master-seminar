\documentclass[11pt]{scrartcl}
\usepackage[utf8]{inputenc}
\usepackage{amsmath}
\usepackage{amsthm}
\usepackage{amssymb}
\usepackage{dsfont}
\usepackage{enumitem}
\usepackage{tikz}
\usepackage{pgfplots}

\theoremstyle{definition}
\newtheorem{example}{Example}[section]

\theoremstyle{definition}
\newtheorem{definition}{Definition}[section]
\renewcommand{\thedefinition}{\arabic{definition}}

\begin{document}

% TODO better title
\title{A bad day of studying is better than a good day of sleeping}
\subtitle{Report on 'How to Revise a Total Preorder' \cite{Booth2011} for seminar 1901 - Representation and processing of uncertain knowledge with logic-based methods}
\author{
	Heltweg, Philip (3230880) \\
	University of Hagen \\
	pheltweg@gmail.com
}
\date{\today}
\maketitle

% TODO abstract
\abstract{Abstract goes here}

\newpage

\tableofcontents

\newpage

% TODO write introduction, make WAY SHORTER, inspire by kai
\section{Introduction}

% goal of paper: interested in function * for getting new tpo

% roten faden des reports beschreiben

\subsection{Research context}
A core process for an agent dealing with uncertain knowledge is updating their worldview when new information becomes available, also called belief change. As an agent can be human or a machine work in belief change has impact on both philosophy and artificial intelligence \cite{Ferme2011}. Different approaches to handling belief change have been discussed with notable approaches including nonmonotonic logic, probabilistic reasoning and belief revision \cite{Darwiche1997}.

In belief revision, belief change is modelled using an operator that produces an updated set of beliefs from the state of an agent and new evidence. Research in this area discusses appropriate formalisations of agent state and constrains on individual or families of operators. Depending on if the suggested operator handles only one new evidence or multiple new evidences the approaches are called one-step or iterated belief revision.

%Consider a knowledge base (KB) represented by a set of sentences in a
%language L. As our perception of the world described by the knowledge base
%changes, the knowledge base must be modified. G~irdenfors [9] distinguishes
%several kinds of modifications. If we simply acquire additional knowledge
%about the world, and the new knowledge does not conflict with the current
%beliefs t of the KB, we expand the KB. If, however, the new knowledge is
%inconsistent with the old beliefs, and we want the KB to be always consistent,
%we must resolve the conflict somehow; this operation will be called revision. A
%* Part of this work was performed while this author was visiting the University of Toronto.
%** This author is partially supported by the Natural Science and Engineering Research Council
%of Canada.
%1 We use the terms belief and knowledge interchangeably in this paper.
%0004-3702/91/$03.50 © 1991 -- Elsevier Science Publishers B.V. All rights reserved 
%264 H. Katsuno, A.O. Mendelzon
%different kind of change occurs when a sentence previously believed becomes
%questionable and has to be given up; the operation that makes this change is a
%contraction

% TODO: discussed in \cite{Katsuno1991} correct? was mentioned in intro of Admissible and Restrained Revision
A common tool used to encode plausibility assumptions of an agent about possible worlds are total preorders (shortened as tpos) \cite{Booth2011}, discussed in \cite{Katsuno1991}. The primary article for this report, 'How to Revise a Total Preorder' discusses how to model the change of these total preorders when new evidence becomes available. Being able to derive a new tpo from additional evidence is an approach to handling multiple revision steps and therefor places the article in the field of iterated belief revision.
%1.2 Ich sehe hier die Gefahr, dass das zu Ausführlich wird.
%Keine große Diskussion von AGM, Darwiche and Pearl und so weiter.
%Keine Beziehung zur Allen und co herstellen ;)
%Kurz und Prägnant präsentieren, was benötgigt wird um zu Verstehen was
%in dem Primärartikel neu ist.

% TODO write formal background
\section{Formal background}

\subsection{Propositional Logic} % Aussagenlogik
Notation follows as used in \cite{Booth2011}: A propositional language $L$ generated from finitely many propositional variables, lower case greek letters represent formulae in $L$ and $\top$, $\bot$ representing tautology and contradiction respectively. $\vdash$ as classical logical consequence, $\equiv$ classical logical equivalence. $Cn$ is used to denote the deductive closure of a formula or set of formulae in $L$.

$W$ is the set of propositional worlds (also called propositional interpretations in classical logic \cite{Kai2020}). Given a $\alpha \in L$ the set of worlds that satisfy $\alpha$ is denoted by $[\alpha]$.

For any set of worlds $S \subseteq W$ of worlds, $Th(S)$ is the set of sentences true in all worlds in $S$.

\subsection{Total preorders}
A total preorder is a binary relation $\leq$ (here over the set of propositional worlds $W$) that is total, reflexive and transitive (i.e. for all $x, y, z \in W$ either $x \leq y$ or $y \leq x$, $x \leq x$ and if $x \leq y$ and $y \leq z$ then $x \leq z$).

The symbol $<$ is used to denote the strict part of $\leq$ while $\sim$ represents the symmetric closure of $\leq$ (i.e. $x \sim y$ iff $x \leq y$ and $y \leq x$).

A helpful visualisation for tpos is shown in \cite{Booth2006}. It uses the fact that tpos can be represented as a linearly ordered set of ranks. Each rank of a tpo $\leq$ is defined as the equivalence classes modulo the symmetric closure of $\leq$: $[[x]]_{\sim} = \{y \mid y \sim x\}$. These equivalence classes are then ordered by the relation $[[x]] \leq [[y]]$ iff $x \leq y$

\begin{example}
    \label{example:example-introduction}
    As an accompanying example consider the following situation, closely aligned to \cite{Booth2011} and \cite{Darwiche1997}: 
    
    Our agent is a judge in a murder trial. John and Mary are suspects. $p$ represents "John is the murderer", $q$ represents "Mary is the murderer" and $r$ represents "the victim is an alien". Possible propositional worlds will be denoted as triplets of 0s and 1s denoting $p$, $q$ and $r$ to be true or false, e.g. the $101$-world stands for John being the murderer and the victim being an alien.
    
    To start the agent believes it is reasonable to assume the murderer acted alone (but are not ruling out both of the accused conspiring). In addition it would be extremely surprising (but not impossible) if the victim was an alien. Tpos are a common tool to handle preference orderings like these.
    
    
    Consider the propositional worlds $W = \{ 000, 001, 010, 011, 100, 101, 110, 111\}$. A tpo representing the judges assumptions would be $\leq$ with $010 \sim 100 < 000 \sim 110 < 011 \sim 101 < 001 \sim 111$. The equivalent representation as a linearly ordered set of ranks is shown in \ref{tab:visualising-a-tpo-example}.

    \begin{table}[h]
         \centering
        \begin{tabular}{llll}
        $R_{1}$                      & $R_{2}$                                                                   & $R_{3}$ & $R_{4}$                      \\ \hline
        \multicolumn{1}{|l|}{\begin{tabular}[c]{@{}l@{}}$010$\\ $100$\end{tabular}} & \multicolumn{1}{l|}{\begin{tabular}[c]{@{}l@{}}$000$\\ $110$\end{tabular}} & \multicolumn{1}{l|}{\begin{tabular}[c]{@{}l@{}}$011$\\ $101$\end{tabular}} &
            \multicolumn{1}{l|}{\begin{tabular}[c]{@{}l@{}}$001$\\ $111$\end{tabular}} \\ \hline
        \end{tabular}
        \caption{Visualising a tpo as a linearly ordered set of ranks as done in \cite{Booth2006}}
        \label{tab:visualising-a-tpo-example}
    \end{table}
\end{example}


% TODO write chapter
\subsection{Belief sets and epistemic states}
A belief set is the deductively closed set of propositions an agent accepts as true at any given point in time \cite{Ferme2011}. AGM theory \cite{Alchourron1985} defines postulates for belief sets and their contraction, revision or expansion with new evidence.
% TODO explain contraction, revision, expansion?
%    As our perception of the world described by the knowledge base
%    changes, the knowledge base must be modified. G~irdenfors [9] distinguishes
%    several kinds of modifications. If we simply acquire additional knowledge
%    about the world, and the new knowledge does not conflict with the current
%    beliefs t of the KB, we expand the KB. If, however, the new knowledge is
%    inconsistent with the old beliefs, and we want the KB to be always consistent,
%    we must resolve the conflict somehow; this operation will be called revision. A
%    * Part of this work was performed while this author was visiting the University of Toronto.
%    ** This author is partially supported by the Natural Science and Engineering Research Council
%    of Canada.
%    1 We use the terms belief and knowledge interchangeably in this paper.
%    0004-3702/91/$03.50 © 1991 -- Elsevier Science Publishers B.V. All rights reserved 
%    264 H. Katsuno, A.O. Mendelzon
%    different kind of change occurs when a sentence previously believed becomes
%    questionable and has to be given up; the operation that makes this change is a
%    contraction. Other kinds of operations are discussed in [14].

Darwiche and Pearl \cite{Darwiche1997} argue that working with belief sets is not expressive enough for satisfying results in iterated belief revision. They make the distinction between propositional beliefs (i.e. beliefs the agent accepts and are part of the belief set) and conditional beliefs (i.e. beliefs the agent is prepared to adopt with new evidence).

While the AGM postulates define restrictions on revising propositional beliefs (a core one being the principle of minimal change), they place little restrictions on preserving conditional beliefs. In their paper Darwiche and Pearl use epistemic states, abstract entities that contain all information an agent needs for it's reasoning, especially their strategy for belief revision. As conditional beliefs are represented by these strategies, it is necessary to define how to modify the strategy itself when encountering multiple new information \cite{Darwiche1997}.

Belief sets can be extracted from epistemic states (the belief set extracted from an epistemic state $\mathbb{E}$ is denoted as $B(\mathbb{E})$). Darwiche and Pearl show that their version of revision of epistemic states can be modelled with a tpo $\leq_{\mathbb{E}}$ associated with $\mathbb{E}$. Extraction of $B(\mathbb{E})$ is done by considering the lowest ranked worlds (i.e. the most plausible interpretations) under $\leq_{\mathbb{E}}$. The set of propositional sentences that holds in all those worlds is assumed to be the belief set of the agent. For notation let $min(\alpha, \leq_{\mathbb{E}})$ denote the set of minimal models for the propositional formula $\alpha$ under $\leq_{\mathbb{E}}$. Then $[B(\mathbb{E})] = min(\top, \leq_{\mathbb{E}})$ is the set of worlds that are models of the belief set $B(\mathbb{E})$ associated with $\mathbb{E}$.

\begin{example}
    Continuing example \ref{example:example-introduction} it is possible to model an epistemic state $\mathbb{E}$ with the tpo $\leq_{\mathbb{E}}$.
    
    $min(\top, \leq_{\mathbb{E}})$ is the set of worlds on the lowest rank, $\{010, 100\}$. A possible belief set is $B(\mathbb{E}) = \{p \vee q, \neg p \vee \neg q, \neg r\}$ as $[B(\mathbb{E})] = \{010, 100\} = min(\top, \leq_{\mathbb{E}})$. Intuitively this makes sense with the initial assumption that John or Mary are the suspects ($p \vee q$), have acted alone ($\neg p \vee \neg q$) and the victim is not an alien ($\neg r$).
\end{example}

% do we need to discuss this in ore detail? faithful assignment here:
%Katsuno and Mendelzon [17] propose that an epistemic state Ψ
%should be equipped with an ordering ≤Ψ of the worlds (interpretations), where the compatibility with Bel(Ψ) is ensured by the socalled faithfulness.
%Definition 1 (Faithful Assignment [17]) A function Ψ 
%→≤Ψ that
%maps each epistemic state to a total preorder on interpretations is
%said to be a faithful assignment if and only if:
%(FA1) if ω1 ∈ Ψ and ω2 ∈ Ψ, then ω1 Ψ ω2
%(FA2) if ω1 ∈ Ψ and ω2 ∈  / Ψ, then ω1 <Ψ ω2
%Intuitively, ≤Ψ orders the worlds by plausibility, such that the minimal worlds with respect to ≤Ψ are the most plausible worlds.

\subsection{DP-AGM postulates}
The postulates first introduced by \cite{Alchourron1985} and reformulated for iterated belief revision on epistemic states by \cite{Darwiche1997} are repeated here. $\mathbb{E}$ denotes an epistemic state, $B(\mathbb{E})$ it's associated belief set, $B(\mathbb{E}) + \alpha$ the expansion of $B(\mathbb{E})$ by $\alpha$ and $\ast$ being a belief change operator on epistemic states. 

\begin{enumerate}[wide=0pt, widest=99,leftmargin=\parindent,label = ($\mathbb{E}\!*\!\arabic*$)]
    \item\label{E1} $\qquad B(\mathbb{E}\ast\alpha) = Cn(B(\mathbb{E}\ast\alpha))$
    \item\label{E2} $\qquad \alpha \in B(\mathbb{E}\ast\alpha)$
    \item\label{E3} $\qquad B(\mathbb{E}\ast\alpha)  \subseteq B(\mathbb{E})+\alpha$
    \item\label{E4} $\qquad \textrm{If } \alpha \notin B(\mathbb{E}) \textrm{ then } B(\mathbb{E}) + \alpha \subseteq B(\mathbb{E} \ast \alpha)$
    \item\label{E5} $\qquad \textrm{If } \mathbb{E} = \mathbb{F} \textrm{ and } \alpha \equiv \beta \textrm{ then } B(\mathbb{E} \ast \alpha) = B(\mathbb{F} \ast \beta)$
    \item\label{E6} $\qquad \bot \in B(\mathbb{E} \ast \alpha) \textrm{ iff } \vdash \neg \alpha$
    \item\label{E7} $\qquad B(\mathbb{E} \ast (\alpha \wedge \beta)) \subseteq B(\mathbb{E} \ast \alpha) + \beta$
    \item\label{E8} $\qquad \textrm{If } \neg \beta \notin B(\mathbb{E} \ast \alpha) \textrm{ then } B(\mathbb{E} \ast \alpha) + \beta \subseteq B(\mathbb{E} \ast (\alpha \wedge \beta))$
\end{enumerate}

To guarantee the ability to extract unique belief sets from epistemic states after a revision by $\alpha$ the Booth and Meyer \cite{Booth2011} modify these postulates by requiring consistent epistemic inputs. That requirement means dropping \ref{E6} and considering only \ref{E1}-\ref{E5} and \ref{E7}-\ref{E8} as DP-AGM.

While DP-AGM define the most plausible worlds after a revision by $\alpha$ (as $min(\alpha, \leq_{\mathbb{E}})$), Darwiche and Pearl also define postulates that restrict the rest of the new ordering. Because the focus of this report is on the revision of tpos the semantic (i.e. in terms of how the ordering of worlds undergoes change) versions are shown here, for more details and sentential versions refer to \cite{Darwiche1997}.

\begin{enumerate}[wide=0pt, widest=99,leftmargin=\parindent,label = (CR$\arabic*$)]
    \item\label{CR1} $\qquad \textrm{If } v\in [\alpha], w \in [\alpha] \textrm{ then } v \leq_{\mathbb{E}} w \textrm{ iff } v \leq_{\mathbb{E\ast\alpha}} w$
    \item\label{CR2} $\qquad \textrm{If } v\in [\neg \alpha], w \in [\neg \alpha] \textrm{ then } v \leq_{\mathbb{E}} w \textrm{ iff } v \leq_{\mathbb{E\ast\alpha}} w$
    \item\label{CR3} $\qquad \textrm{If } v\in [ \alpha], w \in [\neg \alpha] \textrm{ then } v <_{\mathbb{E}} w \textrm{ only if } v <_{\mathbb{E\ast\alpha}} w$
    \item\label{CR4} $\qquad \textrm{If } v\in [ \alpha], w \in [\neg \alpha] \textrm{ then } v \leq_{\mathbb{E}} w \textrm{ only if } v \leq_{\mathbb{E\ast\alpha}} w$
\end{enumerate}

\ref{CR1} and \ref{CR2} mean that the relative ordering of worlds following a revision by $\alpha$ stays the same if the worlds are either both $\alpha$ or $\neg\alpha$ worlds. \ref{CR3} and \ref{CR4} require that $\alpha$-worlds that are strictly/weakly more plausible than $\neg\alpha$-worlds are still strictly/weakly more plausible than them after a $\alpha$-revision.

% TODO: QUESTION: do we need C1-C4 here? 
%\begin{equation*}
%\begin{aligned}
%    (C1) \qquad & \textrm{If } \beta \vdash \alpha \textrm{ then } B(\mathbb{E}\ast\alpha\ast\beta)=B(\mathbb{E}\ast\beta) \\
%    (C2) \qquad & \textrm{If } \beta \vdash \neg  \alpha \textrm{ then } B(\mathbb{E}\ast\alpha\ast\beta)=B(\mathbb{E}\ast\beta) \\
%    (C3) \qquad & \textrm{If } \alpha \in B(\mathbb{E}\ast\beta) \textrm{ then } \alpha \in B(\mathbb{E}\ast\alpha\ast\beta) \\
%    (C4) \qquad & \textrm{If } \neg \alpha \notin B(\mathbb{E}\ast\beta) \textrm{ then } \neg \alpha \notin B(\mathbb{E}\ast\alpha\ast\beta) \\
%\end{aligned}
%\end{equation*}


% TODO write chapter
\section{Additional metadata for tpo revision}
\subsection{Enriching epistemic states}
Based on DP-AGM and the postulates $(CR1)$-$(CR4)$ Booth and Meyer assume a fixed tpo $\leq$ over a set of worlds $W$ that acts as a plausibility ordering. The goal of the paper is the discussion of functions $\ast$ that return a new ordering $\leq_{\alpha}^{\ast}$ for every $\alpha \in L$. Functions $\ast$ that can be used for iterated revision in that way are refered to as revision operators for $\leq$.

Booth and Meyers \cite{Booth2011} approach is to add additional metadata to every world $w \in W$. This new structure is denoted as $W^{\pm} = \{x^{\epsilon} \mid x \in W \textrm{ and } \epsilon \in \{+, -\}\}$. In this notation a world $w \in W$ is represented twice: When new evidence $\alpha$ arrives that makes $w$ more plausible (because $w \in [\alpha]$) the agent can assume it's rank as $w^{+} \in W^{\pm}$. In contrast if the new evidence makes $w$ less plausible ($w \notin [\alpha]$) than $w^{-} \in W^{\pm}$ is $w$'s rank. In their paper Booth and Meyer call the pair $(w^{+}, w^{-})$ the positive/negative representation of the world $w$. It represents an abstract interval representing metadata about plausibility assumptions about $w$.

\subsection{$\leq$-faithful tpos}
For the ordering on $W^{\pm}$ Booth and Meyer suppose an additional relation $\preceq$ over $W^{\pm}$ that is added to the epistemic state of an agent.

\begin{example}
    \label{example:example-stick}
    Recall that tpos can be equivalently thought of as a linearly ordere set of ranks (Example \ref{example:example-introduction}, page \pageref{example:example-introduction}) and therefor a tpo over $W$ can be visualised by ordering every $w \in W$ in a table of ranks (Table \ref{tab:visualising-a-tpo-example}, page \pageref{tab:visualising-a-tpo-example}). To visualise the interval ordering introduced by $\preceq$ Booth and Meyer use a "stick" with the endpoints defined as $w^{+}$/$w^{-}$ respectively, first introduced in \cite{Booth2007} and shown in Figure \ref{fig:example-visualisation-scatterplot}.
        
    \begin{figure}[h]
        \centering
        \begin{tikzpicture}[scale=1.2]
            \begin{axis}[
                  yticklabels={,,$x_{4}$,$x_{3}$,$x_{2}$,$x_{1}$},
                  xticklabels={,,},
                  ytick style={draw=none},
                  xtick style={draw=none},
                  axis line style={draw=none}
            ]
            \addplot[
                scatter,
                scatter src=explicit symbolic,
                mark size=3,
                scatter/classes={
                    empty={mark=o, draw=black},
                    filled={mark=x, draw=black}
                },
                nodes near coords*={\Label},
                visualization depends on={value \thisrow{label} \as \Label}
            ]
            table [meta=class] {
                    x y class label
                    
                    5 1 empty $x_{4}^{+}$
                    7 1 empty $x_{4}^{-}$
                                        
                    3 2 empty $x_{3}^{+}$
                    5 2 empty $x_{3}^{-}$
                    
                    2 3 empty $x_{2}^{+}$
                    4 3 empty $x_{2}^{-}$
                    
                    1 4 empty $x_{1}^{+}$
                    3 4 empty $x_{1}^{-}$
                    };
            \end{axis}
        \end{tikzpicture}
        \label{fig:example-visualisation-scatterplot}
            \caption{Representation of $\preceq$ over $W^{\pm}$ using sticks}
    \end{figure}
    
    In this visualisation the worlds on lower ranks (more on the left) are preferred and therefor assumed to be more plausible. Here for example $x_{1}^{+} \prec x_{2}^{+}$ and $x_{3}^{-} \sim x_{4}^{+}$ Note that while all sticks have the same length in this example that is not required.
\end{example}

To define the relations between $\preceq$ and $\leq$ Booth and Meyer define a list of conditions.

\begin{enumerate}[wide=0pt, widest=99,leftmargin=\parindent,label = ($\preceq\arabic*$)]
    \item\label{PREQ1} $\qquad \preceq \textrm{ is a tpo over } W^{\pm}$
    \item\label{PREQ2} $\qquad x^{+} \preceq y^{+} \textrm{ iff } x \leq y$
    \item\label{PREQ3} $\qquad x^{-} \preceq y^{-} \textrm{ iff } x \leq y$
    \item\label{PREQ4} $\qquad x^{+} \prec x^{-}$
\end{enumerate}

The choice between positive or negative representations of two worlds should be the same as under $\leq$ (due to \ref{PREQ2}, \ref{PREQ3}). According to \ref{PREQ4} there has to be a difference between two ranks of a positive representation and a negative representation of the same world and given the choice between both the positive representation has to be chosen.

\begin{definition}[$\leq$-faithful tpo over $W^{\pm}$ taken directly from \cite{Booth2011}]
\label{definition:faithful-tpo}Let $\preceq \in W^{\pm} \times W^{\pm}$. If $\preceq$ satisfies \ref{PREQ1}-\ref{PREQ4} we say $\preceq$ is a $\leq$-faithful tpo (over $W^{\pm}$).
\end{definition}

Because of \ref{PREQ2} and \ref{PREQ3} it is sufficient to only include $\preceq$ in the epistemic state. The tpo $\leq$ over $W$ can be determined from $\preceq$ by restricting to only $\{ x^{+} \mid x \in W\}$ or $\{ x^{-} \mid x \in W\}$ respectively.

\begin{example}
    \label{example:example-faithful-tpo}
    Is the following tpo $\preceq$ over $W^{\pm}$ a $\leq$-faithful tpo?
        
    \begin{figure}[h]
        \centering
        \begin{tikzpicture}[scale=1]
            \begin{axis}[
                  yticklabels={,,},
                  xticklabels={,,},
                  ytick style={draw=none},
                  xtick style={draw=none},
                  axis line style={draw=none}
            ]
            \addplot[
                scatter,
                scatter src=explicit symbolic,
                mark size=3,
                scatter/classes={
                    empty={mark=o, draw=black},
                    filled={mark=x, draw=black}
                },
                nodes near coords*={\Label},
                visualization depends on={value \thisrow{label} \as \Label}
            ]
            table [meta=class] {
                    x y class label
                                        
                    3 1 empty $x_{3}^{+}$
                    5 1 empty $x_{3}^{-}$
                    
                    2 2 empty $x_{2}^{+}$
                    3 2 empty $x_{2}^{-}$
                    
                    1 3 empty $x_{1}^{+}$
                    4 3 empty $x_{1}^{-}$
                    };
            \end{axis}
        \end{tikzpicture}
    \end{figure}
    
    While \ref{PREQ1} and \ref{PREQ4} are satisfied, $\preceq$ is not a $\leq$-faithful tpo: Due to \ref{PREQ2} and $x_{1}^{+} \prec x_{2}^{+}$ follows $x_{1} < x_{2}$. That means \ref{PREQ3} can not hold as $x_{2}^{-} \prec x_{1}^{-}$ which would require $x_{2} < x_{3}$ to be true.
    
    As demonstrated here while the "sticks" do not need to be the same size (just exist due to \ref{PREQ4}) they need to be the same size for worlds that share a rank for their corresponding representations and can not "cover" other sticks completely to continue to satisfy \ref{PREQ2} and \ref{PREQ3}.
\end{example}

% TODO write chapter
% booth: "on a semantic level, i.e., in terms of how the ordering of individual worlds x, y undergo change"
\section{Tpo revision operators}
Booth and Meyer \cite{Booth2011} aim to discuss tpo revision operators, functions $\ast$ that return a new tpo $\leq_{\alpha}^{\ast}$ for every $\alpha \in L$. To do so they define how to use the additional information from a $\leq$-faithful tpo $\preceq$ over $W^{\pm}$ to create a revision operator $\ast_{\preceq}$.

\begin{definition}[Revision operator $\ast_{\preceq}$ for $\leq$ generated by $\preceq$ from \cite{Booth2011}]
    \label{definition:revision-operator}
    For each $\leq$-faithful tpo $\preceq$ over $W^{\pm}$, refer to $\ast_{\preceq}$ as the revision operator for $\leq$ generated by $\preceq$ defined by:
    
    Set for any $\alpha \in L$ and $x \in W$:
    \begin{equation*}
        r_{\alpha}(x) = \left\{
                    \begin{array}{ll}
                      x^{+} \textrm{ if } x \in [\alpha]\\
                      x^{-} \textrm{ if } x \in [\neg\alpha]
                    \end{array}
                  \right.
    \end{equation*}
    
    The revised tpo $\leq_{\alpha}^{\ast}$ is defined by setting, for each $x, y \in W$,

    \begin{equation*}
        x \leq_{\alpha}^{\ast} y \textrm{ iff } r_{\alpha}(x) \preceq r_{\alpha}(y)
    \end{equation*}
\end{definition}

Intuitively new evidence $\alpha$ makes worlds that satisfy $\alpha$ more plausible and worlds that do not satisfy $\alpha$ less plausible. Therefor the agent associates worlds $w \in [\alpha]$ with their positive representation $w^{+}$ and worlds $w \in [\neg\alpha]$ with their negative representation $w^{-}$.

% visual example with alien -> revise tpo

% show also: new evidence that does not become part of the belief set -> non prio example, explain what it is
\subsection{The problem of just moving things one up}

%\begin{itemize}
%    \item Introduction of $\ast$ as revision operator for $\leq$ and definition of $\ast_{\preceq}$ as example with visualisation.

\subsection{Example operator}
% shortened example from paper page 232 without SPH teil, show that shortened example satisfies the properties later on. USE EXAMPLE from darwiche -> here to illustrate things with guy killing an alien
%TODO this could be OWN contribution!!?

% TODO write chapter
\section{Properties of tpo-revision operators}
%    \item Properties of $\ast_{\preceq}$, their explanation 

%"These
%three rules were considered characteristic of a family of operators called
%admissible revision operators [7]."
% TODO: WHAT ARE ADMISSIBLE OPERATORS
%info here: \cite{Booth2006a}


% THEOREM 1 as axiomatisation of the family of revision operators considered in the paper.

% TODO write chapter
\section{Sentinential view, do we even need this?}
%    \item Explanation of sentinential view (and distinction to semantic level), definition of $\leq' \circ \beta $ and how it relates to conditional beliefs, mention of $\leq' \circ \top $ as belief set. Show of properties for the family of revision operators using sentinential view. Explanation of Disj1/Disj2 as properties that are easy to show in the sentinential view but not in the semantic formulation.
%\end{itemize}
%\textbf{Section Goal:} Definition of $\ast_{\preceq}$ as example revision operator and visualisation of it. Show of axiomatic definition of the family of operators in semantic and sentinential view. Establishing theorem 1.

\section{Summary}

\newpage

\typeout{}
\bibliographystyle{plain}
\bibliography{references}

\newpage

\section*{Erklärung}
Ich erkläre, dass ich die schriftliche Ausarbeitung zum Seminar selbstständig und ohne unzulässige Inanspruchnahme Dritter verfasst habe. Ich habe dabei nur die angegebenen Quellen und Hilfsmittel verwendet und die aus diesen wörtlich oder sinngemäß entnommenen Stellen als solche kenntlich gemacht. Die Versicherung selbstständiger Arbeit gilt auch für enthaltene Zeichnungen, Skizzen oder graphische Darstellungen. Die Ausarbeitung wurde bisher in gleicher oder Ähnlicher Form weder derselben noch einer anderen Prüfungsbehörde vorgelegt und auch nicht veröffentlicht. Mit der Abgabe der elektronischen Fassung der endgültigen Version der Ausarbeitung nehme ich zur Kenntnis, dass diese mit Hilfe eines Plagiatserkennungsdienstes auf enthaltene Plagiate geprüft werden kann und ausschließlich für Prüfungszwecke gespeichert wird.

\end{document}
\documentclass[11pt]{scrartcl}
\usepackage[utf8]{inputenc}
\usepackage{amsmath}
\usepackage{amsthm}
\usepackage{amssymb}
\usepackage{dsfont}


\theoremstyle{definition}
\newtheorem{example}{Example}[section]

\begin{document}

% TODO better title
\title{A bad day of studying is better than a good day of sleeping}
\subtitle{Report on 'How to Revise a Total Preorder' \cite{Booth2011} for seminar 1901 - Representation and processing of uncertain knowledge with logic-based methods}
\author{
	Heltweg, Philip (3230880) \\
	University of Hagen \\
	pheltweg@gmail.com
}
\date{\today}
\maketitle

% TODO abstract
\abstract{Abstract goes here}

\newpage

\tableofcontents

\newpage

% TODO write introduction, make WAY SHORTER, inspire by kai
\section{Introduction}

% goal of paper: interested in function * for getting new tpo

% roten faden des reports beschreiben

\subsection{Research context}
A core process for an agent dealing with uncertain knowledge is updating their worldview when new information becomes available, also called belief change. As an agent can be human or a machine work in belief change has impact on both philosophy and artificial intelligence \cite{Ferme2011}. Different approaches to handling belief change have been discussed with notable approaches including nonmonotonic logic, probabilistic reasoning and belief revision \cite{Darwiche1997}.

In belief revision, belief change is modelled using an operator that produces an updated set of beliefs from the state of an agent and new evidence. Research in this area discusses appropriate formalisations of agent state and constrains on individual or families of operators. Depending on if the suggested operator handles only one new evidence or multiple new evidences the approaches are called one-step or iterated belief revision.

%Consider a knowledge base (KB) represented by a set of sentences in a
%language L. As our perception of the world described by the knowledge base
%changes, the knowledge base must be modified. G~irdenfors [9] distinguishes
%several kinds of modifications. If we simply acquire additional knowledge
%about the world, and the new knowledge does not conflict with the current
%beliefs t of the KB, we expand the KB. If, however, the new knowledge is
%inconsistent with the old beliefs, and we want the KB to be always consistent,
%we must resolve the conflict somehow; this operation will be called revision. A
%* Part of this work was performed while this author was visiting the University of Toronto.
%** This author is partially supported by the Natural Science and Engineering Research Council
%of Canada.
%1 We use the terms belief and knowledge interchangeably in this paper.
%0004-3702/91/$03.50 © 1991 -- Elsevier Science Publishers B.V. All rights reserved 
%264 H. Katsuno, A.O. Mendelzon
%different kind of change occurs when a sentence previously believed becomes
%questionable and has to be given up; the operation that makes this change is a
%contraction

A common tool used to encode plausibility assumptions of an agent about possible worlds are total preorders (shortened as tpos) \cite{Booth2011}. The primary article for this report, 'How to Revise a Total Preorder' discusses how to model the change of these total preorders when new evidence becomes available. Being able to derive a new tpo from additional evidence is an approach to handling multiple revision steps and therefor places the article in the field of iterated belief revision.
%1.2 Ich sehe hier die Gefahr, dass das zu Ausführlich wird.
%Keine große Diskussion von AGM, Darwiche and Pearl und so weiter.
%Keine Beziehung zur Allen und co herstellen ;)
%Kurz und Prägnant präsentieren, was benötgigt wird um zu Verstehen was
%in dem Primärartikel neu ist.

% TODO write formal background
\section{Formal background}

\subsection{Propositional Logic} % Aussagenlogik
Notation follows as used in \cite{Booth2011}: A propositional language $L$ generated from finitely many propositional variables, lower case greek letters represent formulae in $L$ and $\top$, $\bot$ representing tautology and contradiction respectively. $\vdash$ as classical logical consequence, $\equiv$ classical logical equivalence. $Cn$ is used to denote the deductive closure of a formula or set of formulae in $L$.

$W$ is the set of propositional worlds (also called propositional interpretations in classical logic \cite{Kai2020}). Given a $\alpha \in L$ the set of worlds that satisfy $\alpha$ is denoted by $[\alpha]$.

For any set of worlds $S \subseteq W$ of worlds, $Th(S)$ is the set of sentences true in all worlds in $S$.

\subsection{Total preorders}
A total preorder is a binary relation $\leq$ (here over the set of propositional worlds $W$) that is total, reflexive and transitive (i.e. for all $x, y, z \in W$ either $x \leq y$ or $y \leq x$, $x \leq x$ and if $x \leq y$ and $y \leq z$ then $x \leq z$).

The symbol $<$ is used to denote the strict part of $\leq$ while $\sim$ represents the symmetric closure of $\leq$ (i.e. $x \sim y$ iff $x \leq y$ and $y \leq x$).

A helpful visualisation for tpos is shown in \cite{Booth2006}. It uses the fact that tpos can be represented as a linearly ordered set of ranks. Each rank of a tpo $\leq$ is defined as the equivalence classes modulo the symmetric closure of $\leq$: $[[x]]_{\sim} = \{y \vert y \sim x\}$. These equivalence classes are then ordered by the relation $[[x]] \leq [[y]]$ iff $x \leq y$

\begin{example}
    \label{example:example-introduction}
    As an accompanying example consider the following situation, closely aligned to \cite{Booth2011} and \cite{Darwiche1997}: 
    
    Our agent is a judge in a murder trial. John and Mary are suspects. $p$ represents "John is the murderer", $q$ represents "Mary is the murderer" and $r$ represents "the victim is an alien". Possible propositional worlds will be denoted as triplets of 0s and 1s denoting $p$, $q$ and $r$ to be true or false, e.g. the $101$-world stands for John being the murderer and the victim being an alien.
    
    To start the agent believes it is reasonable to assume the murderer acted alone (but are not ruling out both of the accused conspiring). In addition it would be extremely surprising (but not impossible) if the victim was an alien. Tpos are a common tool to handle preference orderings like these.
    
    
    Consider the propositional worlds $W = \{ 000, 001, 010, 011, 100, 101, 110, 111\}$. A tpo representing the judges assumptions would be $\leq$ with $010 \sim 100 < 000 \sim 110 < 011 \sim 101 < 001 \sim 111$. The equivalent representation as a linearly ordered set of ranks is shown in \ref{tab:visualising-a-tpo-example}.

    \begin{table}[h]
         \centering
        \begin{tabular}{llll}
        $R_{1}$                      & $R_{2}$                                                                   & $R_{3}$ & $R_{4}$                      \\ \hline
        \multicolumn{1}{|l|}{\begin{tabular}[c]{@{}l@{}}$010$\\ $100$\end{tabular}} & \multicolumn{1}{l|}{\begin{tabular}[c]{@{}l@{}}$000$\\ $110$\end{tabular}} & \multicolumn{1}{l|}{\begin{tabular}[c]{@{}l@{}}$011$\\ $101$\end{tabular}} &
            \multicolumn{1}{l|}{\begin{tabular}[c]{@{}l@{}}$001$\\ $111$\end{tabular}} \\ \hline
        \end{tabular}
        \caption{Visualising a tpo as a linearly ordered set of ranks as done in \cite{Booth2006}}
        \label{tab:visualising-a-tpo-example}
    \end{table}
\end{example}


% TODO write chapter
\subsection{Belief sets and epistemic states}
A belief set is the deductively closed set of propositions an agent accepts as true at any given point in time \cite{Ferme2011}. AGM theory \cite{Alchourron1985} defines postulates for belief sets and their contraction, revision or expansion with new evidence.

Darwiche and Pearl \cite{Darwiche1997} argue that working with belief sets is not expressive enough for satisfying results in iterated belief revision. They make the distinction between propositional beliefs (i.e. beliefs the agent accepts and are part of the belief set) and conditional beliefs (i.e. beliefs the agent is prepared to adopt with new evidence).

While the AGM postulates define restrictions on revising propositional beliefs (a core one being the principle of minimal change), they place little restrictions on preserving conditional beliefs. In their paper Darwiche and Pearl use epistemic states, abstract entities that contain all information an agent needs for it's reasoning, especially their strategy for belief revision. As conditional beliefs are represented by these strategies, it is necessary to define how to modify the strategy itself when encountering multiple new information \cite{Darwiche1997}.

Belief sets can be extracted from epistemic states (the belief set extracted from an epistemic state $\mathbb{E}$ is denoted as $B(\mathbb{E})$). When modelled with a tpo, extraction is done by considering the lowest ranked worlds (i.e. the most plausible interpretations) under $\leq_{\mathbb{E}}$. The set of propositional sentences that holds in all those worlds is assumed to be the belief set of the agent. For notation let $min(\alpha, \leq_{\mathbb{E}})$ denote the set of minimal models for the propositional formula $\alpha$ under $\leq_{\mathbb{E}}$. Then $[B(\mathbb{E})] = min(\top, \leq_{\mathbb{E}})$ is the set of worlds that are models of the belief set $B(\mathbb{E})$ associated with $\mathbb{E}$.

\begin{example}
    Continuing example \ref{example:example-introduction} it is possible to model an epistemic state $\mathbb{E}$ with the tpo $\leq_{\mathbb{E}}$.
    
    $min(\top, \leq_{\mathbb{E}})$ is the set of worlds on the lowest rank, $\{010, 100\}$. A possible belief set is $B(\mathbb{E}) = \{p \vee q, \neg p \vee \neg q, \neg r\}$ as $[B(\mathbb{E})] = \{010, 100\} = min(\top, \leq_{\mathbb{E}})$. Intuitively this makes sense with the initial assumption that John or Mary are the suspects ($p \vee q$), have acted alone ($\neg p \vee \neg q$) and the victim is not an alien ($\neg r$).
\end{example}

% do we need to discuss this in ore detail? faithful assignment here:
%Katsuno and Mendelzon [17] propose that an epistemic state Ψ
%should be equipped with an ordering ≤Ψ of the worlds (interpretations), where the compatibility with Bel(Ψ) is ensured by the socalled faithfulness.
%Definition 1 (Faithful Assignment [17]) A function Ψ 
%→≤Ψ that
%maps each epistemic state to a total preorder on interpretations is
%said to be a faithful assignment if and only if:
%(FA1) if ω1 ∈ Ψ and ω2 ∈ Ψ, then ω1 Ψ ω2
%(FA2) if ω1 ∈ Ψ and ω2 ∈  / Ψ, then ω1 <Ψ ω2
%Intuitively, ≤Ψ orders the worlds by plausibility, such that the minimal worlds with respect to ≤Ψ are the most plausible worlds.

\subsection{DP-AGM postulates}
%Definition of the DP-AGM postulates and what modifications are made to them for the paper (to always be able to get a belief set from the tpo). Show how a belief set can be extracted from an epistemic state. Introduction of the notation of $\leq$ as tpo, $\ast$ as the revision operator for $\leq$ returning a new tpo $\leq^{\ast}_{\alpha}$.

% TODO write chapter
\section{Enriched epistemic states and $\leq$-faithful tpos}
% recal table 1 with visualisation exampple


%\begin{itemize}
%    \item Notation of the enriched epistemic state $W^{\pm}$, $\preceq$ and conditions on it as well as it's relation to $\leq$. Visual examples using sticks \cite{Booth2011} and ranks (and mentioning that tpos can be represented as linearly ordered set of ranks) \cite{Booth2006}. Definition of a $\leq$-faithful tpo over $W^{\pm}$.
%\end{itemize}
%\textbf{Section Goal:} Notation and introducing the concept of a $\leq$-faithful tpo over $W^{\pm}$ and how to visualise it.

% TODO write chapter
% booth: "on a semantic level, i.e., in terms of how the ordering of individual worlds x, y undergo change"
\section{tpo-revision operators: Revising $\leq$ with $\preceq$}
%\begin{itemize}
%    \item Introduction of $\ast$ as revision operator for $\leq$ and definition of $\ast_{\preceq}$ as example with visualisation.
\subsection{Example operator}
% shortened example from paper page 232 without SPH teil, show that shortened example satisfies the properties later on. USE EXAMPLE from darwiche -> here to illustrate things with guy killing an alien
%TODO this could be OWN contribution!!?

% TODO write chapter
\section{Properties of tpo-revision operators}
%    \item Properties of $\ast_{\preceq}$, their explanation and theorem 1 as axiomatisation of the family of revision operators considered in the paper.

% TODO write chapter
\section{Sentinential view}
%    \item Explanation of sentinential view (and distinction to semantic level), definition of $\leq' \circ \beta $ and how it relates to conditional beliefs, mention of $\leq' \circ \top $ as belief set. Show of properties for the family of revision operators using sentinential view. Explanation of Disj1/Disj2 as properties that are easy to show in the sentinential view but not in the semantic formulation.
%\end{itemize}
%\textbf{Section Goal:} Definition of $\ast_{\preceq}$ as example revision operator and visualisation of it. Show of axiomatic definition of the family of operators in semantic and sentinential view. Establishing theorem 1.

\section{Summary}

\newpage

\typeout{}
\bibliographystyle{plain}
\bibliography{references}

\newpage

\section*{Erklärung}
Ich erkläre, dass ich die schriftliche Ausarbeitung zum Seminar selbstständig und ohne unzulässige Inanspruchnahme Dritter verfasst habe. Ich habe dabei nur die angegebenen Quellen und Hilfsmittel verwendet und die aus diesen wörtlich oder sinngemäß entnommenen Stellen als solche kenntlich gemacht. Die Versicherung selbstständiger Arbeit gilt auch für enthaltene Zeichnungen, Skizzen oder graphische Darstellungen. Die Ausarbeitung wurde bisher in gleicher oder Ähnlicher Form weder derselben noch einer anderen Prüfungsbehörde vorgelegt und auch nicht veröffentlicht. Mit der Abgabe der elektronischen Fassung der endgültigen Version der Ausarbeitung nehme ich zur Kenntnis, dass diese mit Hilfe eines Plagiatserkennungsdienstes auf enthaltene Plagiate geprüft werden kann und ausschließlich für Prüfungszwecke gespeichert wird.

\end{document}
\documentclass[11pt]{scrartcl}
\usepackage[utf8]{inputenc}
\usepackage{amsmath}
\usepackage{amssymb}
\usepackage{dsfont}

\begin{document}

% TODO better title
\title{Report on 'How to Revise a Total Preorder' \cite{Booth2011}}
\subtitle{Seminar 1901 - Representation and processing of uncertain knowledge with logic-based methods}
\author{
	Heltweg, Philip (3230880) \\
	University of Hagen \\
	pheltweg@gmail.com
}
\date{\today}
\maketitle

% TODO abstract
\abstract{Abstract goes here}

\newpage

\tableofcontents

\newpage

% TODO write introduction, make WAY SHORTER, inspire by kai
\section{Introduction}
%1.2 Ich sehe hier die Gefahr, dass das zu Ausführlich wird.
%Keine große Diskussion von AGM, Darwiche and Pearl und so weiter.
%Keine Beziehung zur Allen und co herstellen ;)
%Kurz und Prägnant präsentieren, was benötgigt wird um zu Verstehen was
%in dem Primärartikel neu ist.

\begin{itemize}
    \item Introduction of the research area of belief change and major sub-areas according to \cite{Darwiche1997}: nonmonotonic logic, probabilistic reasoning and belief revision and placing of \cite{Booth2011} in the area of belief revision. Belief revision as operator based approach that concerns itself with the definition of constrains (so called postulates) on operators that transitions between held beliefs when new information arrives \cite{Darwiche1997}.
    \item \cite{Alchourron1985} as seminal paper for one-step belief revision on belief sets. Definitions of belief sets and difference between propositional beliefs and conditional beliefs. Adherence to the principle of minimal change for currently held beliefs only; nearly no constrains on conditional beliefs.
    \item \cite{Grove1988} introducing an ordered worlds / spheres approach instead of working on a set of held beliefs. \cite{Booth2004} understanding preorders as partitions into equivalence classes on ordered worlds (and using multiple orders). Iterated belief change as outlook.
    \item \cite{Darwiche1997} discussion of iterated belief revision and the definition and need for epistemic states instead of belief sets for handling conditional beliefs. total preorders as plausibility orderings to represent conditional beliefs.
    \item \cite{Booth2011} discusses a framework for tpo-revision that enriches an epistemic state with a framework to create a new tpo for a revision step on the basis of interval orderings between positive and negative representations of possible worlds. Motivation of tpo-revision as 'context' as introduced in \cite{Booth2007}.
    \item Therefor \cite{Booth2011} as work in iterated belief revision, closer placement of \cite{Booth2011} as non-priotised revision on the belief set level.
    \item Related areas to the framework established \cite{Booth2011} with preference aggregation and relation to research in interval orderings.
\end{itemize}
\textbf{Section Goal:} Overview of the research area and placement of \cite{Booth2011} as work on iterated belief revision. Rough overview of the background for the work in \cite{Booth2011}. Shared understanding of: belief revision (and the difference between one-step- and iterated belief revision), belief sets and epistemic states, conditional beliefs in contrast to propositional beliefs, tpos as plausibility orderings, motivation and definition of tpo-revision as tool for iterated belief revision. Section will need to be discussed with the seminar on \cite{Darwiche1997} and reduced depending on how much is shown in their presentation.

see \cite{Kai2020} for inspiration

% TODO write formal background
\section{Formal background}
\subsection{subscection for each thing, e.g. total preorder}
\subsection{Notation}

%
%1.3.1/2 Das ist denke ich gut so.

%1.3.3-1.3.4 Ich finde gut das du Dinge ausschließt.
%Ich würde sogar vorschlagen, nur 1.3.1 und 1.3.2 auszuarbeiten und auch
%im Vortrag zu präsentieren. Das Paper hat wirklich sehr viel Inhalt.
%Aber die Schwerpunktsetzung sei aber dir überlassen.
%
%

% TODO expand this into subheadings and agenda
\section{Enriched epistemic state and $\leq$-faithful tpos}
\begin{itemize}
    \item Notation basics.
    \item Definition of the DP-AGM postulates and what modifications are made to them for the paper (to always be able to get a belief set from the tpo). Show how a belief set can be extracted from an epistemic state. Introduction of the notation of $\leq$ as tpo, $\ast$ as the revision operator for $\leq$ returning a new tpo $\leq^{\ast}_{\alpha}$. 
    \item Notation of the enriched epistemic state $W^{\pm}$, $\preceq$ and conditions on it as well as it's relation to $\leq$. Visual examples using sticks \cite{Booth2011} and ranks (and mentioning that tpos can be represented as linearly ordered set of ranks) \cite{Booth2006}. Definition of a $\leq$-faithful tpo over $W^{\pm}$.
\end{itemize}
\textbf{Section Goal:} Notation and introducing the concept of a $\leq$-faithful tpo over $W^{\pm}$ and how to visualise it.

% TODO expand this into subheadings and agenda
\section{tpo-revision operators: Revision of $\leq$ with $\preceq$}
\begin{itemize}
    \item Introduction of $\ast$ as revision operator for $\leq$ and definition of $\ast_{\preceq}$ as example with visualisation.
    \item Properties of $\ast_{\preceq}$, their explanation and theorem 1 as axiomatisation of the family of revision operators considered in the paper.
    \item Explanation of sentinential view (and distinction to semantic level), definition of $\leq' \circ \beta $ and how it relates to conditional beliefs, mention of $\leq' \circ \top $ as belief set. Show of properties for the family of revision operators using sentinential view. Explanation of Disj1/Disj2 as properties that are easy to show in the sentinential view but not in the semantic formulation.
\end{itemize}
\textbf{Section Goal:} Definition of $\ast_{\preceq}$ as example revision operator and visualisation of it. Show of axiomatic definition of the family of operators in semantic and sentinential view. Establishing theorem 1.

\section{Summary}

\newpage

\typeout{}
\bibliographystyle{plain}
\bibliography{references}

\newpage

\section*{Erklärung}
Ich erkläre, dass ich die schriftliche Ausarbeitung zum Seminar selbstständig und ohne unzulässige Inanspruchnahme Dritter verfasst habe. Ich habe dabei nur die angegebenen Quellen und Hilfsmittel verwendet und die aus diesen wörtlich oder sinngemäß entnommenen Stellen als solche kenntlich gemacht. Die Versicherung selbstständiger Arbeit gilt auch für enthaltene Zeichnungen, Skizzen oder graphische Darstellungen. Die Ausarbeitung wurde bisher in gleicher oder Ähnlicher Form weder derselben noch einer anderen Prüfungsbehörde vorgelegt und auch nicht veröffentlicht. Mit der Abgabe der elektronischen Fassung der endgültigen Version der Ausarbeitung nehme ich zur Kenntnis, dass diese mit Hilfe eines Plagiatserkennungsdienstes auf enthaltene Plagiate geprüft werden kann und ausschließlich für Prüfungszwecke gespeichert wird.

\end{document}
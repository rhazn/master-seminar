\documentclass[11pt]{article}
\usepackage[document]{ragged2e}
\usepackage[utf8]{inputenc}
\usepackage[ngerman]{babel}
\usepackage{amsmath}
\usepackage{tikz}
\usepackage{tikz-qtree}
\usepackage[final]{pdfpages}
\usepackage[parfill]{parskip}
\usepackage{listings}

\newcommand\erfequiv{\mathrel{\overset{\makebox[0pt]{\mbox{\normalfont\tiny\sffamily erf}}}{\equiv}}}

\begin{document}

\title{Seminar 1901 - Darstellung und Verarbeitung unsicheren Wissens mit logikbasierten Methoden}
\author{
	Heltweg, Philip
}
\maketitle

\begin{tabular}{l l}
Matrikelnummer: & 3230880\\
\\
Name, Vorname: & Heltweg, Philip\\
\\
Strasse, Nr.: & Kerpener Straße 10\\
\\
Auslandskennzeichen, PLZ, Wohnort: & D 50937 Köln\\
\\
\end{tabular}

\newpage

\section{Goal}
\textbf{Richard Booth and Thomas Andreas Meyer. How to revise a total preorder. J. Philos. Log., 40(2):193–238, 2011.}
Keywords: Epistemic State, Belief Revision, Preference Aggregation, Strict Preference, Social Choice Theory

Totale Präordnungen gelten als die Basisrepräsentation fur iterierte Revision von Wissen. Um auch Prinzipien für das Ändern von totalen Präordnungen zu untersuchen, schlagen Booth und Meyer vor, Intervalle zu ordnen. Diese grundlegende Arbeit zu dieser Idee soll aufbereitet und präsentiert werden.

\begin{itemize}
    \item 'This paper is a combined and extended version of papers which first appeared in the proceedings of KR 2006, the 10th International Conference on Principles of Knowledge Representation and Reasoning [8], and ECSQARU 2007, the 9th European Conference on Symbolic and Quantitative Approaches to Reasoning with Uncertainty [10].'
    \item Booth, R., Meyer, T., \& Wong, K. S. (2006). A bad day surfing is better than a good day working: How to revise a total preorder. In KR (pp. 230–238). AAAI Press.
    \item Booth, R., \& Meyer, T. A. (2007). On the dynamics of total preorders: Revising abstract interval orders. In ECSQARU (pp. 42–53).
\end{itemize}

\section{Concepts}
\begin{itemize}
    \item belief, belief set
    \begin{itemize}
        \item set of held beliefs of an agent, 'true sentences'
    \end{itemize}
    \item belief revision, iterated belief revision, non-prioritised revision
    \begin{itemize}
        \item 'Das JMTS von Doyle berücksichtigt aber nicht nur explizit nichtmonotone Regeln, sondern vollzieht auch einen Wechsel des Wissens- oder Glaubenszustandes (eines maschinellen oder humanen Agenten). Einen solchen Wechsel bezeichnet man als Wissensrevision (belief revision). Heute sind die beiden Bereiche der nichtmonotonen Logiken und der Wissensrevision parallele, aber eng miteinander verwandte Fachgebiete, die sich eines großen Interesses und reger Forschungsaktivitäten erfreuen (vgl. die beiden Übersichtsartikel [Mak94] und [GR94], siehe auch [KI99, KI01]). Die Arbeit von J. Doyle [Doy79] leistete zu beiden einen frühen Beitrag.' - 1845 KE 3/4
        \begin{itemize}
            \item Makinson, D.: General patterns in nonmonotonic reasoning. In: Gabbay, D.M., C.H. Hogger und J.A. Robinson (Herausgeber): Handbook of Logic in Artificial Intelligence and Logic Programming, Band 3, Seiten 35–110. Oxford University Press, 1994.
            \item Gardenfors, P. und H. Rott: Belief Revision. In: Gabbay, D.M., C.H. Hogger und J.A. Robinson (Herausgeber): Handbook of Logic in Artificial Intelligence and Logic Programming, Seiten 35–132. Oxford University Press, 1994.
            \item Kern-Isberner, G.: A unifying framework for symbolic and numerical approaches to nonmonotonic reasoning and belief revision. Fachbereich Informatik der FernUniversität Hagen, 1999. Habilitationsschrift
            \item Kern-Isberner, G.: Conditionals in nonmonotonic reasoning and belief revision. Springer, Lecture Notes in Artificial Intelligence LNAI 2087, 2001.
            \item Doyle, J.: A Truth Maintenance System. Artificial Intelligence, 12:231–272, 1979.
        \end{itemize}
        \item 'Das JTMS-Verfahren so wie es hier vorgestellt wurde, kann allerdings nur angewendet werden, wenn einem TM-Netzwerk Begründungen hinzugefügt werden, nicht aber, wenn Begründungen entfernt oder modifiziert werden. Um diese Fälle zu behandeln, sind andere Methoden der Wissensrevision notwendig.' - 1845 KE 3/4
        \item 'minimal change': 'Auch das DDB-Verfahren kann nur dann erfolgreich durchgeführt werden, wenn Annahmen, also nichtmonoton etablierte Knoten zurückgesetzt werden können. Dadurch wird eine sehr vorsichtige, möglichst geringfügige Änderung ermöglicht. Damit nimmt das JTMS von Doyle ein Paradigma vorweg, das auch heute noch eine zentrale Bedeutung für die Wissensrevision besitzt: Das Paradigma der minimalen Änderung (minimal change.)' - 1845 KE 3/4
        \item 1, 7, 12, 20, 21, 22, 23, 24, 34
        \item \textbf{C. Alchourrón, P. Gärdenfors, and D. Makinson. On the logic of theory change: Partial meet contrac- tion and revision functions. Journal of Symbolic Logic, 50(2):510–530, 1985.}
        \begin{itemize}
            \item 'AGM postulates for belief revision'
            \item theory expansion, revision and contraction with new information, revision as contraction, defining constrains (called postulates) on an operator for theory contraction on new information, operator changes currently held belief set here -> not epistemic states
        \end{itemize}
        \item R. Booth and T. Meyer. Admissible and restrained revision. Journal of Artificial Intelligence Research (JAIR), 26:127–151, 2006.
        \item \textbf{A. Darwiche and J. Pearl. On the logic of iterated belief revision. Artificial Intelligence, 89:1–29, 1997.}
        \begin{itemize}
            \item "reformulated the AGM postulates for belief revision to be compatible with their suggested approach to iterated revision"
            \item works on epistemic states, not belief sets
        \end{itemize}
        \item A. Grove. Two modelings for theory change. Journal of Philosophical Logic, 17:157–170, 1988.
        \begin{itemize}
            \item ordered worlds/spheres approach
        \end{itemize}
        \item S. O. Hansson, E. Fermé, J. Cantwell, and M. Falappa. Credibility-limited revision. Journal of Sym- bolic Logic, 66(4):1581–1596, 2001.
        \item S.O. Hansson. A survey of non-prioritized belief revision. Erkenntnis, 50(2):413–427, 1999.
        \item Y. Jin and M. Thielscher. Iterated belief revision, revised. Artificial Intelligence, 171(1):1–18, 2007.
        \item H. Katsuno and A. O. Mendelzon. Propositional knowledge base revision and minimal change. Artif. Intell., 52(3):263–294, 1991.
        \item A. Nayak, M. Pagnucco, and P. Peppas. Dynamic belief revision operators. Artificial Intelligence,
146:193–228, 2003.
    \end{itemize}
    \item agent
    \item epistemic state of an agent
    \item epistemic entrenchment
    \item sound and complete properties
    \begin{itemize}
        \item Wikipedia: 'Informally, a soundness theorem for a deductive system expresses that all provable sentences are true. Completeness states that all true sentences are provable.' -> Vollständig und Korrekt
    \end{itemize}
    \item total preorder
    \begin{itemize}
        \item Wikipedia: 'In mathematics, especially in order theory, a preorder or quasiorder is a binary relation that is reflexive and transitive. Preorders are more general than equivalence relations and (non-strict) partial orders, both of which are special cases of a preorder. An antisymmetric preorder is a partial order, and a symmetric preorder is an equivalence relation.' + total = for all a, b a <=b or b <= a
        \item transitive, connected relation, unifying semantics booth
        \item unifying semantics booth: preorders partition into equivalence classes
    \end{itemize}
    \item strict preference hierarchies
    \item interval orderings
    \begin{itemize}
        \item 2, 15, 32
        \item J.F. Allen. Maintaining knowledge about temporal intervals. Communications of the ACM, 26(11):832–843, 1983.
        \begin{itemize}
            \item 'such interval orders have been studied in the context of temporal reasoning'
        \end{itemize}
        \item P.C. Fishburn. Interval orders and interval graphs. Wiley, New York, 1985.
        \begin{itemize}
            \item "the question of how our work fits into the more general use of interval orderings"
        \end{itemize}
        \item \textbf{M. Öztürk, A. Tsoukia`s, and P. Vincke. Preference modelling. In Multiple Criteria Decision Analysis: State of the Art Surveys, volume 78, pages 27–71. Springer, 2005.}
        \begin{itemize}
            \item 'such interval orders have been studied in the context of [...] preference modelling'
            \item 'in a similar vein, there seems to be a close connection between our work and the work on preference modelling using interval orderings by Öztürk et al'
        \end{itemize}
    \end{itemize}
    \item preference aggregation
    \begin{itemize}
        \item 3, 19
        \item K. Arrow. Social Choice and Individual Values. John Wiley \& Sons, 1963.
        \item S.M. Glaister. Symmetry and belief revision. Erkenntnis, 49(1):21–56, 1998.
        \begin{itemize}
            \item 'for more discussion on social choice-like conditions and their relevance to tpo-revision we refer the reader...'
        \end{itemize}
    \end{itemize}
\end{itemize}


\section{Ausführliche Inhaltsgliederung}
Bis zu diesem Zeitpunkt erwarten wir von Ihnen eine Kurzfassung Ihres Vortrages, die  Folgendes enthalten sollte: ausführliche Inhaltsgliederung, Angabe der Schwerpunkte, Kommentierung der von Ihnen gewählten Stoffauswahl, Zusammenfassung. Umfang: ca. 4-5 Seiten. Danach werden wir uns wieder an Sie wenden. Bitte schicken Sie uns Ihre Kurzfassung per E-Mail im PDF-Format zu.
\section{Angabe der Schwerpunkte}
Citing entry from my bibliography \cite{kutz_2013}.
\section{Kommentierung der Stoffauswahl}
\section{Zusammenfassung}

\newpage

\bibliographystyle{plain}
\bibliography{references}

\newpage

\section{Erklärung}
Ich erkläre, dass ich die schriftliche Ausarbeitung zum Seminar selbstständig und ohne unzulässige Inanspruchnahme Dritter verfasst habe. Ich habe dabei nur die angegebenen Quellen und Hilfsmittel verwendet und die aus diesen wörtlich oder sinngemäß entnommenen Stellen als solche kenntlich gemacht. Die Versicherung selbstständiger Arbeit gilt auch für enthaltene Zeichnungen, Skizzen oder graphische Darstellungen. Die Ausarbeitung wurde bisher in gleicher oder Ähnlicher Form weder derselben noch einer anderen Prüfungsbehörde vorgelegt und auch nicht veröffentlicht. Mit der Abgabe der elektronischen Fassung der endgültigen Version der Ausarbeitung nehme ich zur Kenntnis, dass diese mit Hilfe eines Plagiatserkennungsdienstes auf enthaltene Plagiate geprüft werden kann und ausschließlich für Prüfungszwecke gespeichert wird.

\end{document}
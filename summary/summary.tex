\documentclass[11pt]{article}
\usepackage[document]{ragged2e}
\usepackage[utf8]{inputenc}
\usepackage{amsmath}
\usepackage[parfill]{parskip}
\usepackage{listings}
\usepackage[acronym]{glossaries}

\newcommand\erfequiv{\mathrel{\overset{\makebox[0pt]{\mbox{\normalfont\tiny\sffamily erf}}}{\equiv}}}

\makeglossaries
\newglossaryentry{principle of minimal change}
{
    name=principle of minimal change,
    description={todo \cite{Darwiche1997}}
}
\newglossaryentry{total preorder}
{
    name=total preorder,
    description={todo}
}
\newglossaryentry{non-priotised revision}
{
    name=non-priotised revision,
    description={revision inputs are not necessarily elements of the belief set associated to the revised preorder \cite{Booth2006}}
}
\newglossaryentry{iterated belief revision}
{
    name=iterated belief revision,
    description={'the sequential revision of beliefs in response to a string of observations' \cite{Darwiche1997}}
}
\newacronym{tpos}{tpos}{Total preorders}

\begin{document}

\title{Seminar 1901 - Darstellung und Verarbeitung unsicheren Wissens mit logikbasierten Methoden}
\author{
	Heltweg, Philip (3230880) \\
	pheltweg@gmail.com
}
\maketitle

\newpage

\tableofcontents

\newpage

\section{Presentation agenda}
\subsection{Introduction}
\subsubsection{Preview}
\subsection{Research Context and definitions}
\begin{itemize}
    \item Introduction of the research area of belief change and major sub-areas according to \cite{Darwiche1997}: nonmonotonic logic, probabilistic reasoning and belief revision and placing of \cite{Booth2011} in the area of belief revision.
    \item Belief revision as operator based approach that concerns itself with the definition of constrains (so called postulates) on operators that transitions between held beliefs when new information arrives \cite{Darwiche1997}.
    \item \cite{Alchourron1985} as seminal paper for one-step belief revision on belief sets. Definitions of belief sets and difference between propositional beliefs and conditional beliefs. Adherence to the principle of minimal change for currently held beliefs only nearly no constrains on conditional beliefs.
    \item \cite{Darwiche1997} discussion of iterated belief revision and the definition and need for epistemic states instead of belief sets for handling conditional beliefs. Introduction of tpos as plausibility orderings to represent conditional beliefs.
    \item \cite{Booth2011} discusses a framework for tpo-revision that enriches an epistemic state with a framework to create a new tpo for a revision step on the basis of interval orderings between positive and negative representations of possible worlds. Motivation of tpo-revision as "context" as introduced in \cite{Booth2007}.
    \item Therefor \cite{Booth2011} as work in iterated belief revision, closer placement of \cite{Booth2011} as non-priotised revision on the belief set level.
    \item Related areas to the framework established \cite{Booth2011} with preference aggregation and relation to research in interval orderings.
\end{itemize}
\textbf{Section Goal:} Overview of the research area and placement of \cite{Booth2011} as work on iterated belief revision. Rough overview of the background for the work in \cite{Booth2011}. Shared understanding of: belief revision (and the difference between one-step- and iterated belief revision), belief sets and epistemic states, conditional beliefs in contrast to propositional beliefs, tpos as plausibility orderings, motivation and definition of tpo-revision as tool for iterated belief revision.

\subsection{Summary}
\subsection{Outlook}
\subsection{Sources}

\section{Research context}
\begin{itemize}
    \item Main Area: Belief change
    \item Sub-areas according to \cite{Darwiche1997}
    \begin{itemize}
        \item Belief revision ('belief changes can be characterized by a set of constraints (called postulates) on an operator $\circ$ which modifies the set $\psi$ of currently held beliefs to produce a new belief set $\psi \circ \mu$ implying the new information $\mu$', 'operator based framework')
        \item Nonmonotonic logic ('belief change is viewed as a byproduct of extending a database containing new facts in accordance with a set of extension-construction rules called defaults') 
        \item Probabilistic reasoning ('belief change is viewed as a byproduct of conditioning a probability function (or some qualitative abstraction thereof) on new evidence, according to bayes rule')
    \end{itemize}
    \item \cite{Booth2011} as work in belief revision
    \item 'On the belief set level: \gls{non-priotised revision} in contrast to most works in iterated belief change that are given in priotised revision' \cite{Booth2011}
    \item In the area of \gls{iterated belief revision} \cite{Booth2011}
\end{itemize}

\section{Literature}
\begin{itemize}
    \item The 'AGM Paper' \cite{Alchourron1985} introduced 'AGM postulates' for belief revision. Works on a \textbf{belief set} of currently held beliefs and with one step revision. Does 'minimal change' for beliefs but not for conditional beliefs.
    \item \cite{Darwiche1997} introduces iterated belief revision and reformulates the AGM postulates to be compatible with their approach. Works on 'epistemic states' of agents, not belief sets. Expands the operator based approach to those states.
    \item \cite{Grove1988} introduces an ordered worlds / spheres approach instead of working on a set of held beliefs.
    \item \cite{Booth2004} preorders as partitions into equivalence classes, ordered worlds with multiple orders $\leq$, $\preceq$, iterated belief change as outlook.
    \item The original research paper \cite{Booth2011} which is a combined and extended version of \cite{Booth2006} and \cite{Booth2007}. Most papers use a one step revision operator. Here: Framework for multi-step revision by introducing meta info in the epistemic state for revision over a preference of worlds, then revision over that meta info structure. Using \gls{tpos}
\end{itemize}

\newpage

\setglossarystyle{listgroup}
\printglossaries

\newpage

\bibliographystyle{alpha}
\bibliography{references}

\newpage

\section{Erklärung}
Ich erkläre, dass ich die schriftliche Ausarbeitung zum Seminar selbstständig und ohne unzulässige Inanspruchnahme Dritter verfasst habe. Ich habe dabei nur die angegebenen Quellen und Hilfsmittel verwendet und die aus diesen wörtlich oder sinngemäß entnommenen Stellen als solche kenntlich gemacht. Die Versicherung selbstständiger Arbeit gilt auch für enthaltene Zeichnungen, Skizzen oder graphische Darstellungen. Die Ausarbeitung wurde bisher in gleicher oder Ähnlicher Form weder derselben noch einer anderen Prüfungsbehörde vorgelegt und auch nicht veröffentlicht. Mit der Abgabe der elektronischen Fassung der endgültigen Version der Ausarbeitung nehme ich zur Kenntnis, dass diese mit Hilfe eines Plagiatserkennungsdienstes auf enthaltene Plagiate geprüft werden kann und ausschließlich für Prüfungszwecke gespeichert wird.

\end{document}